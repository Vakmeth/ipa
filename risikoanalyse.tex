\storeareas\riskvalues
\KOMAoptions{paper=a3, paper=landscape, DIV=current}
\areaset
  {\dimexpr\the\paperwidth-1cm\relax}
  {\dimexpr\the\paperheight-5.5cm\relax}
\recalctypearea

\chapter{Risikoanalyse}
\begin{table}[H]
  \begin{tabular}{ |C{0.01\textwidth}|C{0.1\textwidth}|C{0.1\textwidth}|C{0.02\textwidth}|C{0.02\textwidth}|C{0.03\textwidth}|C{0.1\textwidth}|C{0.2\textwidth}|C{0.02\textwidth}|C{0.02\textwidth}|C{0.03\textwidth}|C{0.1\textwidth}| }
      \hline
      \multirow{2}*{Nr} & \multirow{2}*{Risikobeschreibung} & \multirow{2}*{Auswirkung} & \multicolumn{4}{|l|}{Vor Massnahme}& \multirow{2}*{Massnahmen} & \multicolumn{4}{|l|}{Nach Massnahme} \\
      \cline{4-7} \cline{9-12}&&& W & S & Risiko & Handlungsweise &&  W & S & Risiko & Handlungsweise \\
      \hline 
      1 & TODO: Name & TODO: Beschreibung & TODO: Auswirkung & TODO: Auswirkung & \cellcolor{yellow}TODO: Risiko & TODO: Handlungsweise 
      & TODO: Massnahme & TODO: Auswirkung & TODO: Auswirkung & \cellcolor{yellow}TODO: Risiko & TODO: Handlungsweise \\
      \hline
  \end{tabular}
  \caption{Risikoanalyse}
\end{table}

\textbf{Schadensausmass:} \\
S1 = führt zu keiner Abwertung \\
S2 = geringe Abwertung bis 1,0 Notenpunkte \\
S3 = hohe Abwertung über 1,0 Notenpunkte \\
S4 = führt zu Nichtbestehen \\

\textbf{Eintrittswahrscheinlichkeit:} \\
W1 = unvorstellbar \\
W2 = unwahrscheinlich \\
W3 = eher vorstellbar \\
W4 = vorstellbar \\
W5 = Eintreffen hoch \\

\restoregeometry
\riskvalues



