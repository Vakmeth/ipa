\section{Tag 9: 18.03}
\begin{table}[H]
    \begin{tabular}{|L{0.4\textwidth}|C{60pt}|C{60pt}|C{60pt}|}
        \hline
        \rowcolor{puzzleblue}\color{white}Tätigkeiten & \color{white}Beteiligte \color{white}Personen & \color{white}Aufwand Geplant (std) & \color{white}Aufwand Effektiv (std) \\
        \hline
         Turbostreams implementieren & Marc Egli & 0 & 0.75 \\
        \hline
        Daily abhalten & Marc Egli & 0.25 & 0.25 \\
        \hline
        Globale Bedingungen bearbeiten & Marc Egli & 0 & 1 \\
        \hline
        Feature Tests schreiben & Marc Egli & 3.25 & 4 \\
        \hline
        Manuelle Tests durchführen & Marc Egli & 1 & 1 \\
        \hline
        KI-Dokumentation schreiben & Marc Egli & 2 & 0.5 \\
        \hline
        Instruktion schreiben und durchführen & Marc Egli & 1 & 0 \\
        \hline
        Arbeitsjournal schreiben & Marc Egli & 0.25 & 0.25 \\
        \hline
        \textbf{Total} &  & 7.75  & 7.75 \\
        \hline
    \end{tabular}
    \caption{Tätigkeiten Tag 9}
\end{table}

\subsection*{Tagesablauf}
Heute begann ich mit den letzten Anpassungen an den Turbostreams. Diese waren recht schnell erledigt und ich konnte alle restlichen funktionalen und nicht
funktionalen Anforderungen meines Anforderungskatalogs abdecken. Dann starteten wir wie immer mit dem Daily um 09:00 Uhr.
Im Daily sprach ich zu Beginn eine Anpassung am Mockup an, welche mir im Schlaf über das Wochenende zugekommen ist. Es wäre für den Benutzer doch viel sinnvoller, 
wenn er die Eingaben direkt im Formular machen könnte, ohne zuerst auf ein Bearbeiten-Symbol zu klicken. So müsste er nur auf die Benutzerschnittstelle der Filterung navigieren
und anschliessend die neuen Werte eingeben. Daniel Illi meinte zu der Idee, das er sie als Stakeholder so akzeptiert. 
Des weiteren habe ich mit Daniel Illi einen Termin für das Durchführen der Instruktion vereinbart. 

Nach dem Daily begann ich mit dem Bearbeiten der globalen Bedingungen. Ich bekam diese schnell durch und startete direkt mit den Featuretests.
Während dem Schreiben der Tests hatte ich stets mein Testkonzept zur Hand. Bei dem Testfall Nr. 17 merkte ich, dass dieser nicht nötig ist, da bereits der Testfall Nr. 12
den Zugriff des Benutzers auf die Personenfilterung abdeckt. Falls der Zugriff nicht gegeben ist, kann der Benutzer auch den Personenfilter in der Benutzerschnittstelle nicht einsehen.
Dementsprechend habe ich den Testfall verworfen und dies im Testprotokoll so dokumentiert. 

Später kam bei mir dann noch eine Frage bezüglich dem Stubben von Umgebungsvariablen auf. Ein Fehleralert wird im Hitobito nur angezeigt,
falls die Umgebung auf Produktion gestellt ist. Da mir nicht klar war, wie ich diese Umgebungsvariable stubben kann, schrieb ich Daniel Illi auf dem Puzzle Chat und bat ihn um 
Hilfe. Da er im Moment noch ein Meeting hatte, schrieb er mir, dass er später persönlich vorbei komme. Mit Daniel Illi konnte ich das Problem anschliessend lösen und 
die restlichen Tests implementieren.

Zuletzt führte ich die manuellen Tests durch und begann mit der Dokumentation der Verwendung von KI in dieser IPA.

\subsection*{Hilfestellungen}
\begin{itemize}
    \item Daniel Illi und Robin Steiner: Mockup Änderung und Terminvereinbarung Instruktion
    \item Daniel Illi: Stubbing von Umgebungsvariablen
\end{itemize}

\subsection*{Reflexion}

\subsubsection*{Was lief gut}
Trotz des Verzuges im Zeitplan konnte ich die Implementation meiner IPA abschliessen. Der Code steht und funktioniert auch, was durch meine
Tests bewiesen wird. Dies erleichtert mich sehr, da ich von diesem Teil der IPA am meisten Respekt hatte. Zudem bin ich im Zeitplan wieder auf Kurs
und denke das ich das Ende wie geplant. 

\subsubsection*{Was lief weniger gut}
Ein Teil von mir hatte gehofft, mit den Featuretests schnell fertig zu werden, um noch mehr Zeit zu gewinnen. Allerdings, hatte ich während des Schreibens der Tests
immer wieder Konzentrationsprobleme. Das Testen ist für mich schlichtweg nicht der schöne Teil der Entwickelerarbeit. Umso erleichterter bin ich nun, 
da alle Testfälle nach Testkonzept umgesetzt werden konnten. 

\subsubsection*{Meine Erkenntnisse von heute}
Wie ich realisiert habe, wurde der Zeitblock für die User-Story ``Turbostreams implementieren'' im Zeitplan massiv zu gross. Im Nachhinein, wäre es 
besser gewesen ich hätte diese User-Story in weitere, kleinere User Stories unterteilt. Zum Beispiel hätte ich 
eine User-Story für die Turbostreams und dann für jede Überabeitung der Benutzerschnittstelle eines Filterkriteriums eine User Story machen können.
Meine Erkenntniss dadurch ist, nicht probieren allzu viele Aufgaben in einem Ticket zusammenzufassen. Besser ist es, kleine Blöcke zu machen.
Das macht die Abnahme der Akzeptanzkriterien im Sprint Review auch einfacher.

\subsection*{Nächste Schritte}
Morgen werde ich zuerst die Dokumentation zur Verwendung von KI in meiner IPA fertigstellen.
Danach bereite ich die Instruktion vor. Wenn beides erledigt ist, mache ich den Sprintabschluss für die Umsetzungsphase. 
Ich werde danach das Sprintplanning für den letzten Sprint durchführen und alle Tickets für die Detailarbeiten an der Dokumentation verfassen.
Das Ziel für morgen ist es, den Umsetzungssprint abzuschliessen und alles bereit für den letzten Sprint der Finalisierung zu haben.

\pagebreak
