\section{Tag 1: 04.03.2025}
\begin{table}[H]
    \begin{tabular}{|L{0.4\textwidth}|C{60pt}|C{60pt}|C{60pt}|}
        \hline
        \rowcolor{puzzleblue}\color{white}Tätigkeiten & \color{white}Beteiligte \color{white}Personen & \color{white}Aufwand geplant (std) & \color{white}Aufwand effektiv (std) \\
        \hline
        Raum einrichten, Kriterien aufhängen & Marc Egli & 1 & 1 \\
        \hline
        Zeitplan erstellen & Marc Egli & 1 & 1 \\
        \hline
        Sprint Planning durchführen & Marc Egli & 1 & 1.5 \\
        \hline
        Task / Standards beschreiben & Marc Egli & 1 & 1 \\
        \hline
        Projektvorgehensmethode beschreiben & Marc Egli & 2 & 1.5 \\
        \hline
        Risikoanalyse beschreiben & Marc Egli & 2 & 2.75 \\
        \hline
        Arbeitsjournal schreiben & Marc Egli & 0.25 & 0.5 \\
        \hline
        \textbf{Total} &  & 8.25 & 9.25 \\
        \hline
    \end{tabular}
    \caption{Tätigkeiten Tag 1}
\end{table}

\subsection*{Tagesablauf}
Ich startet heute Morgen um 07.45 Uhr mit der IPA. Als Erstes begann ich damit, den Raum einzurichten, was bedeutet: 
Docking Station anschliessen, Wasser bereitstellen und alle Kriterien meiner IPA aufhängen. Danach habe ich alle Kriterien mit verschiedenen
Farben unterteilt: Blau steht für Kriterien, welche über die gesamte IPA hinweg zählen, Rosa für Kriterien, welche in der Umsetzung zu beachten sind und 
Gelb für Kriterien, welche die Dokumentation betreffen. Als ich mit der Zimmereinrichtung fertig war, startete ich direkt mit dem Zeitplan. Ich passte das Template,
welches ich vorbereitet habe, auf die Dauer der IPA an und machte alles bereit um die ersten User Stories einzutragen.  Nachdem der Zeitplan fertig war, startete ich das Sprint Planning.
Darin organisierte ich als Erstes das Daily mit meiner verantwortlichen Fachkraft und meiner zusätzlichen verantwortlichen Fachkraft. Das Daily setzte ich auf
09:00 Uhr morgens an. 

Später im Planning, habe ich alle nötigen User Stories für den kommenden Sprint definiert und diese anschliessend in den Zeitplan mit der dazugehörigen Schätzung übertragen.
Auf der Uhr war nun schon 11:15 Uhr. Ich startete den ersten Teil des Beschriebes der Aufgabenstellung und der Firmenstandards und ging danach in den Mittag.

Nach dem Mittag beendete ich den Beschrieb der Aufgabenstellung und der Firmenstandards und begann mit der Projektvorgehensmethode. Hier kam ich überraschend schnell durch und konnte so
nach 1.5 Stunden die Risikoanalyse beginnen, an welcher ich bis kurz vor dem Schluss des Tages, 17:30 gearbeitet habe. Beim Erstellen der Risikoanalyse bemerkte ich, dass ich noch Fragen
zum Berechtigungskonzept in Hitobito hatte. Dementsprechend ging ich zu Niklas Jäggi, welcher mir dann das Konzept erklärte.
Ganz am Ende schrieb ich dann noch das Arbeitsjournal.

\subsection*{Hilfestellungen}
\begin{itemize}
    \item Niklas Jäggi: Erklärung des Berechtigungsaufbaus in Hitobito
\end{itemize}

\subsection*{Reflexion}

\subsubsection*{Was funktonierte gut}
Der Einstieg funktionierte meiner Meinung nach sehr gut. Ich kam schnell voran und konnte die ersten paar Teile der Dokumentation
beschreiben. Sogar das erste Kriterium, A11 (Projektaufbauorganisation) konnte ich schon abschliessen, was mich sehr motiviert hat. 

\subsubsection*{Was funktionierte weniger gut}
Obwohl ich schnell vorwärts kam, habe ich heute dennoch den geplanten Aufwand um 1/4-Stunde überschossen. Hier muss ich aufpassen, dass ich unbedingt früher anfange das
Arbeitsjournal zu schreiben. Zusätzlich hatte ich beim Sprint Planning ein Problem mit dem Erstellen eines Issue-Templates. Ich hatte mich spontan dazu entschieden,
dass es sehr hilfreich wäre, ein Template zu haben, in welchem man neue Issues während der IPA erfassen kann und so nicht alles immer neu machen muss. Allerdings hatte ich noch 
nie ein solches Template erstellt, weswegen das Planning dann auch eine 1/2-Stunde mehr Zeit in Anspruch nahm als geplant.

\subsubsection*{Meine heutigen Erkenntnisse}
Nicht allzu viel viel Zeit mit Themen verlieren, in welchen ich wenig Erfahrung habe. Besser wäre es gewesen mit dem Issue-Template zu warten
und dann in einem Daily danach zu fragen. Dennoch kann ich nun das Wissen um die Erstellung dieses Templates schon als ersten Erfolg in
dieser IPA verbuchen.

\subsection*{Nächste Schritte}
Morgen werde ich eine Zusammenfassung der Risikoanalyse verfassen, um das Kriterium G5 (Risikoanalyse und Sicherheitsmassnahmen) abzuschliessen.
Danach werde ich weiter am Board arbeiten, dass heisst, als nächstes die Sektionen Versionierung und Backup in der Dokumentation beschreiben.
Zusätzlich findet am Morgen noch der erste Expertenbesuch statt, welcher mir perfekt dient, um meinen vorbereiteten Fragenkatalog abzuarbeiten. 
Hier werde ich sicher Fragen zu organisatorischen Bereichen der IPA stellen, wie dem Zeitplan, Diagrammen oder dem Code-Anhang.

\pagebreak
