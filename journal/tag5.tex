\section{Tag 5: 11.03.2025}
\begin{table}[H]
    \begin{tabular}{|L{0.4\textwidth}|C{60pt}|C{60pt}|C{60pt}|}
        \hline
        \rowcolor{puzzleblue}\color{white}Tätigkeiten & \color{white}Beteiligte \color{white}Personen & \color{white}Aufwand Geplant (std) & \color{white}Aufwand Effektiv (std) \\
        \hline
         Testkonzept abschliessen & Marc Egli & 0 & 0.5 \\
        \hline
        Sprintabschluss machen & Marc Egli & 1 & 0.5 \\
        \hline
        Daily & Marc Egli, Robin Steiner, Daniel Illi & 0.25 & 0.5 \\
        \hline
        Sprint Planning durchführen & Marc Egli & 1 & 2 \\
        \hline
        Tasks und Standards beschreiben & Marc Egli & 2 & 0.75 \\
        \hline
        Overview implementieren & Marc Egli & 2 & 1.5 \\
        \hline
        Add-Dropdown implementieren & Marc Egli & 1.25 & 1.75 \\
        \hline
        Arbeitsjournal schreiben & Marc Egli & 0.25 & 0.25 \\
        \textbf{Total} &  & 7.75 & 7.75 \\
        \hline
    \end{tabular}
    \caption{Tätigkeiten Tag 5}
\end{table}

\subsection*{Tagesablauf}
Den Morgen begann ich mit dem Fertigstellen des Testkonzeptes. Die Schnittstellen welche ich noch überarbeiten wollte, werde
ich in einer Story unterbringen, in welcher ich die nötigen Endpoints auch gerade implementiere, so kann ich beides gleichzeitig erledigen 
ohne viel Zeit zu verlieren. Nach der Fertigstellung des Testkonzeptes schloss ich den Sprint ab. Nahezu alle stories konnte ich auf ``Done''
schieben. Eine Story jedoch nicht, da ich zwei Teile in der Dokumentation noch nicht beschrieben habe. Somit viel diese User Story zurück ins 
``Refinement''. Nach dem Sprintabschluss fand das Daily statt.

Im Daily sprach ich zum einen das Kriterium zu den Anforderungen an. Dort ist beschrieben das die Anforderungen lösungsneutral seien müssen, jedoch
erfüllen meine Anforderungen dies nicht, da ich ein Mockup zur Vefügung habe und somit schon eine Lösung vorgegeben habe. Robin Steiner
riet mir dieses Problem morgen mit meinem Hauptexperten zu besprechen. Danach fragte ich meine verantwortlichen Fachkräte ob ich Sätze in der Dokumentation
mit ``In diesem Abschnitt wird XY erklärt'' beginnen darf. Die Antwort: Ja, wenn es sinnvoll eingesetzt wird. Die nächste meiner Fragen richtete sich an den 
Anforderungskatalog. Hier wird im Kriterium G6 beschrieben dass die Sicherheitsmassnahmen hinterlegt werden müssen.
Da ich bereits alle Sicherheitsrisiken und die Massnahmen dafür erfasst habe, fragte ich ob es reiche, wenn ich die Sicherheitsmassnahmen im Anforderungskatalog als
Link hinterlegen könne. Robin Steiner und Daniel Illi bestätigten mir dies. Als nächstes habe ich mit meine verantwortlichen Fachkräften Diagrammstandards beschrieben.
hierbei ging es darum, ob ich Standards verwenden soll oder nicht. Die Antwort: Morgen mit dem Hauptexperten klärten.
Des weiteren fragte ich ob es möglich sei mit einem Test mehrere Anforderungen abzudecken. Daniel Illi antwortete darauf, dass
dies möglich sei, wenn es sich um einen Feature Test handelt. Falls es ein Unit-Test ist, dürfe er nur eine einzelne Funktion abdecken. 

Zum Schluss besprach ich im Team den überarbeiten Anforderungskatalog wie es vom Kriterium G6 Punkt vier gefordert wird. Die Änderungen welche meine Stakeholder 
vorgeschlagen haben, nahm ich auf und werde diese im Anforderungskatalog nachführen. Änderungen bezüglich des Katalogs umfassten vor allem die Formulierung,
welche laut Robin Steiner noch etwas ``zu holprig'' seien.

Nach dem Daily führte ich das Sprint Planning für den nächsten Sprint durch und Plant die Tasks der Umsetzung.
Den Nachmittag verbrachte ich bis zum Schluss mit der Implementierung der ersten User Stories. Hierbei freute ich mich sehr, da 
ich endlich mit der Umsetzung staten konnte. Während des Entwickelns kamen mir Fragen zum Styling und welches Standard Template bei einem 
GET request auf eine Ressource zurückgegeben wird. Gemäss Kriterium A10, Punkt 4 habe ich in unserem internen Firmenchat diese Fragen an Daniel Illi gestellt,
um während dem Warten auf eine Antwort spetitif weiterarbeiten zu können. Bei der Implementation des Dropdown hatte ich zuletzt den Fehler, 
dass das ganze Array der Filterkriterien im Stil ``[Tags, Attribute, Rollen, Qualifikationen]'' angezeigt wurde. Diesen Fehler muss ich morgen noch beheben.

Zum Schluss des Tages, schrieb ich das Arbeitsjournal.

\subsection*{Hilfestellungen}
\begin{itemize}
    \item Robin Steiner, Daniel Illi: Alle Fragen welche gemäss Tagesablauf dokumentiert wurden
\end{itemize}

\subsection*{Reflexion}

\subsubsection*{Was lief gut}
Heute hatte ich das Gefühl das ich sehr schnell voran kam. Ich konnte die meisten Tasks wie geplant abhandeln und bin 
momentan mehr oder weniger im Zeitplan. Ausserdem hatte ich viel mehr Freude an der Arbeit, da ich num endlich zum Entwickeln kam.

\subsubsection*{Was lief weniger gut}
Ich hatte zu Beginn Probleme in den ``Entwicklermodus'' zu kommen und mich schnell im Code zurecht zu finden.
Dies löste sich aber schnell und ich konnte die erste User Story bezüglich der Implementation umsetzen. Zudem kommte, dass ich das 
gesetzte Tagesziel erreicht habe und mit der Umsetzung starten konnte.

\subsubsection*{Meine Erkenntnisse von heute}
Wichtig ist es, dass ich meine Zeiten immer sauber raportiere ansonsten habe ich Probleme. Ich merkte dies im Verlaufe des heutigen Tages, als ich mich plötzlich gefragt habe ``Was genau hast du jetzt die letzte Stunde gemacht?''.
Als Gegenmassnahme habe ich gemerkt, dass es mir sehr hilft,
wenn ich immer nach Abschluss einer User Story den Zeitblock auf Papier notiere und dann direkt in den Zeitplan einschreibe. So habe ich am Ende 
des Tages eine saubere Übersicht über meine Zeiten und kann diese dementsprechend im Arbeitsjournal eintragen.

\subsection*{Nächste Schritte}
Morgen werde ich den Fehler im Dropdown der Filterkriterien beheben und anschliessend mit der Umsetzung weiterverfahren. Unter anderem sollen laut
meiner Planung die Endpoints für die jeweiligen Partials implementiert werden.

\pagebreak
