\section{Tag 12: 21.03.2025}
\begin{table}[H]
    \begin{tabular}{|L{0.4\textwidth}|C{60pt}|C{60pt}|C{60pt}|}
        \hline
        \rowcolor{puzzleblue}\color{white}Tätigkeiten & \color{white}Beteiligte \color{white}Personen & \color{white}Aufwand geplant (std) & \color{white}Aufwand effektiv (std) \\
        Daily abhalten & Marc Egli, Robin Steiner, Daniel Illi & 0.25 & 0.25 \\
        \hline
        Instruktion anpassen & Marc Egli & 1 & 1 \\
        \hline
        Zeitplan finalisieren & Marc Egli & 1 & 1 \\
        \hline  
        Abgabe und letzte Änderungen machen & Marc Egli & 2 & 2 \\
        \hline
        Sprintabschluss machen & Marc Egli & 0.5 & 0.5 \\ 
        \hline
        Arbeitsjournal schreiben & Marc Egli & 0.25 & 0.25 \\
        \hline
        \textbf{Total} &  & 4.75 & 4.75 \\
        \hline
    \end{tabular}
    \caption{Tätigkeiten Tag 12}
\end{table}

\subsection*{Tagesablauf}

Heute war ein stressiger Tag. Die ganze Abgabe und Korrektur machte mich nervöser als gedacht. Ich begann den Tag damit, letzte Anpassungen an der Instruktion zu machen.
Danach finalisierte ich den Zeitplan und machte mich an die letzten Änderungen. Stand jetzt, sind noch zwei Stunden bis zur Abgabe vorhanden. Ich werde in diesen 
Stunden so viel korrigieren wie möglich und das ganz Dokument so gut es geht auf Fehler prüfen. Im Daily heute fragte ich meine verantwortlichen Fachkräfte, ob 
ich den Code im Anhang formattieren müsste. Sie meinten an sich nicht, wenn es jedoch nicht zu viel Aufwand macht dann schon. Ich werde den Code in den nächsten Stunden 
so gut formattieren wie es die Zeit zulässt.

Nachdem ich dieses Journal fertiggeschrieben habe, werde ich die erste Abgabe auf das PKORG laden. So verhindere ich, dass bei einem Notfall, eine verspätete Abgabe entsteht. 
Diesen Abzug gilt es unbedingt zu verhindern. 

\subsection*{Hilfestellungen}
\begin{itemize}
    \item Daniel Illi und Robin Steiner: Nachfrage zur Formatierung von Code im Anhang
\end{itemize}

\subsection*{Reflexion}

\subsubsection*{Was funktionierte gut}
Ich konnte die Arbeit finalisieren und bin insgesamt sehr zufrieden damit. Ich habe alle nötigen Informationen im Bericht dokumentiert 
und freue mich nun auch ein bisschen die Arbeit endlich abgeben zu können.

\subsubsection*{Was funktionierte weniger gut}
Nebst dem Stress und dem Druck von heute, lief alles tiptop. Allerdings habe ich das Gefühl, das mich genau dieser 
Stress besser abliefern lässt, ich durch erhöhte Motivation und schnelleres Arbeiten bemerken konnte.

\subsubsection*{Meine heutigen Erkenntnisse}
Meine Erkenntnis von heute: Die IPA war eine sehr aufregende, stressige aber auch lehrreiche Erfahrung. Ich konnte meine Arbeit nun abschliessen 
und werde den Freitagnachmittag geniessen.

\subsection*{Nächste Schritte}
Die Abgabe werde ich voraussichtlich in zwei oder eineinhalb Stunden signieren. Danach, gilt es in den kommenden Wochen die Präsentation und das 
Fachgespräch vorzubereiten.

\pagebreak
