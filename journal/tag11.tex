\section{Tag 11: 21.03.2025}
\begin{table}[H]
    \begin{tabular}{|L{0.4\textwidth}|C{60pt}|C{60pt}|C{60pt}|}
        \hline
        \rowcolor{puzzleblue}\color{white}Tätigkeiten & \color{white}Beteiligte \color{white}Personen & \color{white}Aufwand geplant (std) & \color{white}Aufwand effektiv (std) \\
        \hline
         Kurzbericht der IPA verfassen & Marc Egli & 1 & 1 \\
        \hline
        Dokumentation gegenlesen und korrigieren & Marc Egli & 4 & 4 \\
        \hline
        Glossar und Abkürzungen definieren & Marc Egli & 1 & 1 \\
        \hline
        Code Anhang generieren & Marc Egli & 1 & 1 \\
        \hline
        Zeitplan finalisieren & Marc Egli & 1 & 1 \\
        \hline
        Arbeitsjournal schreiben & Marc Egli & 0.25 & 0.25 \\
        \hline
        \textbf{Total} &  & 8.25 & 8.25 \\
        \hline
    \end{tabular}
    \caption{Tätigkeiten Tag 11}
\end{table}

\subsection*{Tagesablauf}
Heute ging es an die Finalisierung der Arbeit. Ich verfasste als erstes den Kurzbericht der IPA und begann danach 
die Dokumentation gegenzulesen. Danach erstellte ich das Glossar mit allen Abkürzungen und Definitionen. Im Daily hatte ich keine 
besonderen Fragen an meine verantwortlichen Fachkräfte, ich präsentierte lediglich den Stand der Arbeit.

Gegen den Mittag führten Daniel Illi und ich dann die Instruktion durch. Dies sehr gut und ich konnte mit der Instruktion nachweisen, dass 
die Benutzerschnittstelle intuitiv erstellt wurde. Falls der Benutzer nicht weiter weiss, hat sich die erstellte Instruktion als praktisch und um Erfolg führend
erwiesen. Am Ende des Tags wertete ich die Ergebnisse der Instruktion aus und finalisierte den Zeitplan. Danach schrieb ich das Arbeitsjournal.

\subsection*{Hilfestellungen}
\begin{itemize}
    \item Keine
\end{itemize}

\subsection*{Reflexion}

\subsubsection*{Was funktionierte gut}
Heute kam ich schnell voran und konnte viele Fehler in der Dokumentation beheben. Zusätzlich sind nun praktisch alle Kapitel darin beschrieben, nur noch der letzte 
Sprintabschluss muss gemacht werden, danach ist alles fertig. Ich war während des Tages immer wieder ein bisschen nervös, da die Abgabe nun schon vor der Tür steht.
Ich denke aber,  dass ich nun nur noch einen Schlussprint hinlegen muss und dann alles gut kommt.

\subsubsection*{Was funktionierte weniger gut}
Alles funktonierte heute gut, es gab keine unvorhergesehenen Herausforderungen oder Problem.

\subsubsection*{Meine heutigen Erkenntnisse}
Ich muss unbedingt viel Zeit in das Gegenlesen stecken und absichern, dass alles perfekt beschrieben ist. Es ist mir sehr wichtig das
diese Arbeit gut herauskommt, dementsprechend muss ich versuchen die Dokumentation fehlerlos zu gestalten.

\subsection*{Nächste Schritte}
Morgen werde ich den Zeitplan fertig finalisieren, den Sprintabschluss machen und eine frühe Abgabe wagen. Dies aus dem Grund, das wenn etwas schief geht,
gleichwohl schon eine Version meiner Dokumentation im PKORG abgelegt ist. Den Rest der Zeit werde ich mit Korrekturen an der Dokumentation verbringen, bis am Mittag
die Durchführung der IPA endet.

\pagebreak
