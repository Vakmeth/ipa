\section{Tag 2: 05.03.2025}
\begin{table}[H]
    \begin{tabular}{|L{0.4\textwidth}|C{60pt}|C{60pt}|C{60pt}|}
        \hline
        \rowcolor{puzzleblue}\color{white}Tätigkeiten & \color{white}Beteiligte \color{white}Personen & \color{white}Aufwand Geplant (std) & \color{white}Aufwand Effektiv (std) \\
         \hline
         Daily Meeting & Marc Egli, Robin Steiner, Daniel Illi & 0.25 & 0.25 \\
         \hline
         Expertenbesuch und Vorbereitung des Besuches & Marc Egli, Robin Steiner, Daniel Illi, Lorenz Müller & 1.5 & 2.25 \\
         \hline
         Versionierung und Backup & Marc Egli & 2 & 1.25 \\
         \hline
         Arbeitsjournal & Marc Egli & 0.25 & 0.25 \\
        \hline
        \textbf{Total} &  & 4 & 4 \\
        \hline
    \end{tabular}
    \caption{Tätigkeiten Tag 2}
\end{table}

\subsection*{Tagesablauf}
Am morgen startete ich mit der Vorbereitung des Expertenbesuches. Danach fand unmittelbar das Daily statt.
Im Daily präsentierte ich den verantwortlichen Fachkräften den Stand der IPA. Danach stellte ich eine Rückfrage an Daniel Illi bezüglich des Berechtigungskonzeptes, da
ich 100\% sicher sein wollte, das die Informationen welche ich von Niklas Jäggi bezogen haben stimmen. Die Nachfrage ergab, dass das Berechtigungskonzept stimme, jedoch ein Diagram
dies noch falsch abbildete. Ich notierte mir somit die Änderung welche ich an diesem Diagramm noch machen muss und schloss das Daily ab.
Nebst der Nachfrage zum Berechtigungskonzept, fragte ich ob es in Ordnung sei, wenn ich reale Personen-Namen in einem Diagramm verwende. Z.B. Heinz
statt User 1. Hierzu bekam ich das OK meiner verantwortlichen Fachckräte.

Nach dem Daily fand dann der Expertenbesuch statt. Das Sitzungsprotokoll hierzu habe ich im Anhang hinterlegt. Der Besuch lief gut und ich konnte vieles daraus mitnehmen
unter anderem das ich eine Person für das Gegenlesen auwählen darf (ohne diese angeben zu müssen). Ausserdem bekam ich weitere Inputs betreffend dem Zeitplan und meiner Kriterien.

Nach dem Expertenbesuch begann ich mit der Sektion zur Versionierung und der Backup-Strategie meiner IPA. Diese konnte ich zeitig abschliessen und danach das Arbeitsjournal verfassen.

\subsection*{Hilfestellungen}
\begin{itemize}
    \item Daniel Illi: Nachfrage bezüglich des Berechtigungskonzeptes
    \item Robin Steiner und Daniel Illi: Nachfrage der Verwendung von Echtnamen in Diagrammen
\end{itemize}

\subsection*{Reflexion}

\subsubsection*{Was lief gut}
Der Tag heute war vor allem dem Expertenbesuch gewidmet, welcher ich sehr positiv fand. Obwohl es noch ein paar Anpassungen zu
machen gibt, so denke ich das durch die Hinweise meines Hauptexperten diese IPA gut herauskommen wird. Wichtig ist jetzt, dass ich das 
Protokoll für diesen Besuch verfasse und alle gewünschten Änderungen umsetze. 

\subsubsection*{Was lief weniger gut}
Heute hatte ich den Eindruck das nichts negativ gelaufen ist. Obwohl es ein paar Fehleinschätzungen im
Zeitplan gab, bin ich dennoch immer noch auf Kurs.

\subsubsection*{Meine Erkenntnisse von heute}
Alle Erkenntnisse welche ich im Sitzungsprotokoll vermerkt habe. Ausserdem nehme ich noch einen Satz meines Hauptexperten
mit: "Nachvollziehbarkeit ist wichtig". Für mich heisst das, alles so klar wie möglich in der Dokumentation zu beschreiben und 
stets einen Blick auf die Kriterien zu werfen.

\subsection*{Nächste Schritte}
Morgen werde ich damit verbringen die Analyse und die Bedürfniserhebung vorzubereiten. Zusätzlich werde
ich die gesammelten Änderungsvorschläge meines Hauptexperten in einem Sitzungsprotokoll aufführen und im Anhang hinterlegen.
Die gewünschten Änderungen werde ich dann ebenso direkt umsetzen.

\pagebreak
