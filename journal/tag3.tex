\section{Tag 3: 06.03.2025}
\begin{table}[H]
    \begin{tabular}{|L{0.4\textwidth}|C{60pt}|C{60pt}|C{60pt}|}
        \hline
        \rowcolor{puzzleblue}\color{white}Tätigkeiten & \color{white}Beteiligte \color{white}Personen & \color{white}Aufwand geplant (std) & \color{white}Aufwand effektiv (std) \\
        \hline
        Daily abhalten & Marc Egli, Daniel Illi, Robin Steiner & 0.25 & 0.25 \\
        \hline
        Bedürfniserhebung durchführen & Marc Egli, Thomas Ellenberg & 4 & 4 \\
        \hline
        Analyse durchführen und beschreiben & Marc Egli & 4 & 4 \\
        \hline
        Arbeitsjournals schreiben & Marc Egli & 0.25 & 0.25 \\
        \textbf{Total} &   & 8.5 & 8.5 \\
        \hline
    \end{tabular}
    \caption{Tätigkeiten Tag 3}
\end{table}

\subsection*{Tagesablauf}
Heute startete ich mit der Vorbereitung der Bedürfniserhebung. Ich habe diese als erste Aufgabe am Morgen geplant,
um später für das Meeting mit Thomas Ellenberg um 13:00 Uhr, vorbereitet zu sein. Im Daily um 09:00 Uhr, präsentierte ich wie gewohnt den 
aktuellen Stand der IPA meinen verantwortlichen Fachkräften. Im Daily fragte mich Daniel Illi nach den besprochenen Inhalten des gestrigen Dailies,
woraufhin ich ihm meine protokollierten Informationen mündlich weitergab. Danach merkte Robin an, dass ich vergessen hatte, meinen Hauptexperten nach der 
Verwendun von Echtnamen in Diagrammen zu fragen. Dies werde ich morgen per Mail nachholen. Zusätzlich wies mich Robin Steiner darauf hin, dass die ursprüngliche
Scrum Definition vorgesehen hätte, alle drei Sprints grob zu planen und danach die Detailplanung in den Sprint Plannings zu erledigen. Da ich ein anderes Vorgehen gewählt
habe, habe ich mir notiert diese Abweichung noch im Abschnitt der Projektvorgehensmethode zu dokumentieren. Abschliessend zum Daily, fragte ich Daniel Illi ob er gerade
wisse, ob wir eine Anleitung für die Filterung von Personen im Hitobito haben. Er antwortete, dass ihm das nicht bekannt sei, ich dies jedoch im Benutzerhandbuch nachschlagen
könne. Diese Information habe ich benötigt, um eine Wahl für die Bedürfniserhebungsmethode zu treffen und somit die Dokumentenanalyse
auszuschliessen.

Nachdem ich nach dem Daily die Bedürfniserhebung vorbereitet hatte,
startete ich in die Analyse. Dort begann ich damit, die Ist-Situation aufzunehmen. Gegen den Mittag wurde ich mit dem Beschrieb der 
Ist-Situation fertig. Nach dem Mittag startete ich direkt mit der Bedürfniserhebung mit Thomas Ellenberg als Testperson. Nachdem ich das Interview 
durchgeführt hatte, sammelte ich alle Resultate und definierte daraus die Bedürfnisse. Danach merkte ich, dass ich einen Fehler bezüglich den commit Messages gemacht habe.
Laut Firmenstandard muss dort stets das Ticket selbst auch angegeben werden, falls eines besteht. Diverse Dokumentierungsaufgaben, welche eich am zweiten Tag erledigt habe,
haben auf meinem Github Projects Board ein Ticket. Ich hatte dies allerdings nicht in der commit Message hinterlegt. Um dies zu korrigieren, habe ich in den Github Docs nachgeschlagen,
um zu sehen, wie ich ältere commit Messages bearbeite. Darunter fand ich eine Anleitung, welche es mir möglich gemacht hat, die fehlenden User-Story-Referenzen in der 
Message zu hinterlegen. Allerdings wurden alle Commits beim Push auf den heutigen Tag gelegt. Dies entspricht nicht der Ursprungsverfassung dieser Commits.
Um dieses Problem zu lösen werde ich es morgen im Daily vorbringen und allenfalls meinen Hauptexperten um Rat fragen.

Gegen 14:00 Uhr konnte ich die Bearbeitung der Bedürfnisse
abschliessen und begann die Soll-Situation in der Analyse zu beschreiben. Die Soll-Situation konnte ich gegeg 16:20 Uhr abschliessen anschliessend begann ich,
die Anforderungen aus der gemachten Risikoanalyse und den Bedürfnissen zu erstellen. Zum Schluss des Tages, verfasste ich wie gewohnt das Arbeitsjournal.

\subsection*{Hilfestellungen}
\begin{itemize}
    \item Daniel Illi: Nachfrage der Benutzerdokumentation der Personenfilterung
\end{itemize}

\subsection*{Reflexion}

\subsubsection*{Was funktionierte gut}
Ich hatte heute das Gefühlt, dass ich mich viel besser konzentrieren konnte. Ich konnte viel länger gezielt arbeiten
und hatte den Eindruck, dass ich schnell vorankomme. Die Bedürfniserhebung war zudem sehr interessant. Der vereinbarte Termin
mit Thomas Ellenberg als Testperson, fand wie geplant statt und ich konnte alle Fragen, wie vorbereitet, stellen.

\subsubsection*{Was funktionierte weniger gut}
Obwohl ich schnell vorankam, konnte ich die Analyse nicht vollständig abschliessen. Es bleiben noch die Anforderungen und die 
Dokumentation der Rahmenbedingungen offen. Ich rechne mit +/- einer Stunde zusätzlichen Aufwand für die Analyse.

\subsubsection*{Meine heutigen Erkenntnisse}
Wichtig ist es früh meinen Fokus zu finden und unnötige Details zu vernachlässigen. Ich sollte mich während dem Arbeiten stets an die Kriterien
als Leitfaden halten. Alles was ich erarbeite und nicht in den Kriterien festgehalten ist, wird auch keine Punkte geben und entspricht somit
dem Aufwand für nichts.

\subsection*{Nächste Schritte}
Der nächste Schritt wird morgen der Abschluss der Analyse. Dies sollte in den ersten zwei Stunden passieren, mehr darf ich unbedingt nicht
überziehen. Danach beginne ich mit dem Entwurf welcher Systemkonzept, Testkonzept, etc. umfasst. Ziel ist es, den Entwurf morgen abschliessen zu können,
um dann am kommenden Dienstag die Umsetzung zu starten.

\pagebreak
