\section{Tag 7: 13.03.2025}
\begin{table}[H]
    \begin{tabular}{|L{0.4\textwidth}|C{60pt}|C{60pt}|C{60pt}|}
        \hline
        \rowcolor{puzzleblue}\color{white}Tätigkeiten & \color{white}Beteiligte \color{white}Personen & \color{white}Aufwand geplant (std) & \color{white}Aufwand effektiv (std) \\
        Endpoints dokumentieren & Marc Egli & 0 & 0.25 \\
        \hline
        Add-Dropdown implementieren & Marc Egli & 4.25 & 3.5 \\
        \hline
        Daily abhalten & Marc Egli & 0.25 & 0.25 \\
        \hline
        Turbo Streams implementieren & Marc Egli & 3 & 3.5 \\
        \hline
        Arbeitsjournal schreiben & 0.25 & 0.25 \\
        \textbf{Total} &  & 7.75 & 7.75 \\
        \hline
    \end{tabular}
    \caption{Tätigkeiten Tag 7}
\end{table}

\subsection*{Tagesablauf}
Wie geplant beginn ich den Tag mit der Dokumentation der Endpoints. Nach der Dokumentation begann ich damit die ersten Turbostreams zu implementieren.
Danach startete das Daily. Im Daily präsentierte ich den Stand der IPA. Danach stellte ich eine Frage zum Kriterium A10. Darin ist beschrieben, dass die Standpunkte der
anderen Projektmitglieder erkannt wurden. Meine Frage war, ob das Arbeitsjournal als Nachweis reicht, da ich hier schon alle meine Fragen und Antworten der verantwortlichen 
Fachkräfte festhalte. Robin Steiner und Daniel Illi bestätigten mir dies. Nach dem Daily startet ich wieder voll mit der Implementation durch. 
Am morgen und frühen Nachmittag hatte ich das Gefühl sehr schnell voranzukommen. Ich konnte das Dropdown effizient umsetzen und habe sogar noch ein paar Änderungen am Konzept gemacht,
welche die Implementation sauberer machen. Gegen den späten Nachmittag hingegen nahm meine Konzentration ab. Ich denke, dass es an einem Bug lag, den ich sehr lange
bearbeitet habe. Es ging spezifisch um die Tags, welche sich nicht sauber anzeigen liessen. Um nicht noch mehr Zeit hier zu verschwende, entschied ich mich
das Problem morgen im Daily mit meinen verantwortlichen Fachkräften zu besprechen und mit den anderen Partials weiterzumachen. 

Während des Implementierens hatte ich stets den Anforderungskatalog offen um sicherzustellen, dass ich alle Anforderungen abdecken kann.
Wie ich bemerkte habe ich die funktionale Anforderung 2 an mein Produkt durch die Implementation des Hinzufüge-Buttons abgedeckt. Auch die Funktionale Anforderung 4 konnte
ich durch die Implementation der Overview abschliessen. 

Ausserdem ist mir eine Bemerkung zum Mockup bezüglich des Filterkriteriums ``Felder'' aufgefallen. Da ich sowieso einen modularen Aufbau des Filters geplant habe, könnte ich die 
Komponente lediglich mit der neuen Übersicht anordnen und am visuellen nichts verändern, da das Filterkriterium ``Felder'' bereits sehr modular aufgebaut ist. Ich werde diese Erkenntniss ebenfalls
morgen im Daily mit meinen verantwortlichen Fachkräften besprechen. 

Gegen Ende des Tages schrieb ich wie gewohnt das Arbeitsjournal.


\subsection*{Hilfestellungen}
\begin{itemize}
    \item Daniel Illi und Robin Steiner: Nachfrage bezüglich Kriterium A10 wie dokumentiert in Tätigkeiten
\end{itemize}

\subsection*{Reflexion}

\subsubsection*{Was funktionierte gut}
Der Fokus heute morgen war exzellent. Ich konnte viel implementieren und die ersten Anforderungen abschliessen. 
Zudem habe ich gemerkt, dass wenn ich endlich einen Bug behoben habe oder ich eine Hürde überwinden konnte, ich mehr Motivation
entwickelte was mir sehr beim fokussieren half. Zudem konnte ich das gestern gesetzte Tagesziel erfüllen und die neue Benutzerschnittstelle ermöglicht das Hinzufügen
von Filterkriterien über das Hinzufüge-Dropdown.

\subsubsection*{Was funktionierte weniger gut}
Am Nachmittag verschwendete ich definitiv zu viel Zeit mit dem Bug beim Filterkriterium der ``Tags''. Je länger ich an einem solchen Bug arbeite, desto mehr
verliere ich die Konzentration und Motivation. Zusätzlich verschwende ich hier wertvolle Zeit, welche ich für andere Baustellen hätte verwenden können.

\subsubsection*{Meine heutigen Erkenntnisse}
Um dem Problem des Zeitaufwandes für Bugs entgegenzukommen definiere ich einen Zeitblock von 20 Minuten welche ich für eine Hürde aufwenden darf. 
Wenn ich diese Zeithürde überschreibe, lasse ich das Problem vorerst liegen und spreche es im Daily an. Falls mich die Hürde an der Weiterarbeit hindert, 
gehe ich umgehend zu meiner verantwortlichen Fachkraft und hole mir Hilfe um das Problem lösen zu können.

\subsection*{Nächste Schritte}
Morgen werde ich die letzten Anpassungen in den Partials der Qualifikationen und Rollen machen. Ziel ist es, dass
alle Partials überarbeite wurden und dem erstellten Mockup entsprechen.

\pagebreak
