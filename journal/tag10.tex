\section{Tag 10: 19.03.2025}
\begin{table}[H]
    \begin{tabular}{|L{0.4\textwidth}|C{60pt}|C{60pt}|C{60pt}|}
        \hline
        \rowcolor{puzzleblue}\color{white}Tätigkeiten & \color{white}Beteiligte \color{white}Personen & \color{white}Aufwand Geplant (std) & \color{white}Aufwand Effektiv (std) \\
        \hline
        Daily abhalten  & Marc Egli & 0.25 & 0.25 \\ 
        \hline
        KI-Dokumentation schreiben & Marc Egli & 0 & 1.5 \\
        \hline
        Instruktion schreiben und durchführen & Marc Egli & 2 & 1.5 \\
        \hline
        Sprintabschluss machen & Marc Egli & 1 & 0.75 \\
        \hline
        Sprint Planning durchführen & Marc Egli & 1 & 1 \\
        \hline
        Arbeitsjournal schreiben & Marc Egli & 0.25 & 0.25 \\
        \hline
        \textbf{Total} &  & 4.5 &  5.25 \\
        \hline
    \end{tabular}
    \caption{Tätigkeiten Tag 10}
\end{table}

\subsection*{Tagesablauf}
Heute startete ich den morgen mit dem Abschluss der KI-Dokumentation. Danach bereitete ich die Instruktion für Donnerstag vor.
Im Daily heute hatte ich keine Fragen, alles geht nun gegen ein Ende zu und es geht darum noch die letzten Feinschliffe an der Arbeit 
zu machen.
Nach dem Daily und dem Vorbereiten der Instruktion habe ich den Sprintabschluss gemacht. Hier habe ich gemerkt, dass ich sehr knapp in der Zeit für den
heutigen Tag bin. Um gleichwohl, das gestrige Ziel zu erreichen, habe ich mich dazu entschieden, das Ziel von heute zu erfüllen und dafür etwas von meiner
Restzeit zu verwenden. Somit kann ich morgen dann optimal in den Finalisationssprint starten. Nach dem Planning habe ich 
abschliessend das Arbeitsjournal geschrieben. 

\subsection*{Hilfestellungen}
\begin{itemize}
    \item Keine
\end{itemize}

\subsection*{Reflexion}

\subsubsection*{Was lief gut}
Obwohl ich mehr Zeit dafür benötigte konnte ich das gestern gesetzte Ziel erreichen. Der Umsetzungssprint ist
abgeschlossen und alle User Stories sind bereite um im letzten Sprint bearbeitet zu werden.

\subsubsection*{Was lief weniger gut}
Heute habe ich keinen Punkt den ich hier aufführen könnte. Auch wenn ich mehr Zeit benötigte, sehe ich dies
nicht als direkten Rückschlag im Projekt, da ich dafür die nötige Zeit eingeplant habe.

\subsubsection*{Meine Erkenntnisse von heute}
Nicht gegen Ende schlapp machen. Am Freitagmorgen werde ich die Arbeit abgeben, darum jetzt noch Endspurt und 
Zähne zusammenbeissen um die Arbeit fertigzubringen.

\subsection*{Nächste Schritte}
Morgen startet der Finalisationssprint. Wichtig ist, dass ich hier spetitif durchgehe und Fehler in der Dokumentation korrigiern kann.
Ausserdem werde ich die Instruktion mit Daniel Illi durchführen, was es mir ermöglicht den letzten Teil der Dokumentation abzuschliessen und das 
Kriterium A15 zu erfüllen.

\pagebreak
