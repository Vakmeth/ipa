\section{Tag 6: 12.03.2025}
\begin{table}[H]
    \begin{tabular}{|L{0.4\textwidth}|C{60pt}|C{60pt}|C{60pt}|}
        \hline
        \rowcolor{puzzleblue}\color{white}Tätigkeiten & \color{white}Beteiligte \color{white}Personen & \color{white}Aufwand Geplant (std) & \color{white}Aufwand Effektiv (std) \\
        \hline
         Add-Dropdown Fehler beheben & Marc Egli & 0 & 0.5 \\
        \hline
        Daily & Marc Egli, Robin Steiner & 0.25 & 0.25 \\
        \hline
        2. Expertenbesuch & Marc Egli, Robin Steiner, Daniel Illi & 0.75 & 0.75 \\
        \hline
        Endpoints implementieren & Marc Egli & 2 & 2.25 \\
        \hline
        Arbeitsjournal schreiben & Marc Egli & 0.25 & 0.25 \\
        \textbf{Total} &  & Gesamtstunden soll & Gesamtstunden ist \\
        \hline
    \end{tabular}
    \caption{Tätigkeiten Tag 6}
\end{table}

\subsection*{Tagesablauf}
Wie geplant habe ich heute morgen zuerst den Fehler im Dropdown behoben. Der Fehler lag in etwas sehr offensichtlichem: Wenn man in Ruby das ``=''-Zeichen 
für Logik verwendet, wird der entsprechende Output auch in der View als Text angezeigt. Da ich eine For-Each-Schleife benutzt habe, um alle Filterkriterien im Dropdown
als Optionen anzuzeigen und dabei das ``=''-Zeichen verwendet habe, wurde am Schluss das ganze Array als Text angezeigt. Um dieses Fehlverhalten zu beheben gilt es nur das ``='' zu einem 
``-'' zu wechseln. 

Nachdem ich diesen Fehler behoben habe, führten wir das Daily durch. Anwesend waren nur Robin Steiner und ich, da ich die Terminänderung des Dailies aufgrund des Expertenbesuches zu spät
mitgeteilt habe und er dadurch nicht am Daily teilnehmen konnte. Dies war jedoch nicht weiter schlimm, da ich für das heutige Daily keine Fragen notiert habe.

Im Daily präsentierte ich Robin Steiner der Stand der IPA und bestätigte das ich alle Punkte welche mir beim letzten Expertenbesuch aufgezeigt wurden, behoben habe.

Nach dem Daily machte ich alles bereit für die Implementation der Endpoints. Später startete der Expertenbesuch. Die besprochenen Angelegenheiten sind wie zuvor in einem Sitzungsprotokoll
einzusehen. Nachdem Expertenbesuch führte ich meine Arbeit an den Endpoints fort. Ich realisierte, dass ich anders als geplant nur einen Endpoint, anstatt einen Endpoint pro Filterkriterium brauche.
Die Realisation ist so sogar noch besser: Es kann der Route schlichtweg das Filterkriterium mitgegeben werden und die Action des Controllers selbst entscheidet dann über den Parameter der den ich der Route
mitgebe, welches Partial gerendert werden muss. So müssen nicht vier verschiedene Endpoints mit eigenen Routen eingesetzt werden, welche am Schluss trotzdem die gleiche Aufgabe haben.

Durch die Änderung an den Anforderungen und weil ich zwischendurch ein Problem damit hatte, die nötigen Berechtigungen an die Route zu vergeben, hat die Umsetzung der Endpoints länger gedauert
als gedacht.

Zum Schluss des Tages schrieb ich wie gewohnt das Arbeitsjournal.

\subsection*{Hilfestellungen}
\begin{itemize}
    \item Keine
\end{itemize}

\subsection*{Reflexion}

\subsubsection*{Was lief gut}
Ich war heute sehr froh, konnte ich den Fehler im Dropdown spetitif beheben. Umso besser war es, dass ich alle Endpoints in einem zusammenfassen konnte. 
So habe ich nun eine saubere Ausgangslage, um die Turbo Streams zu erfassen. Ebenso hatte ich das Gefühl, dass der Expertenbesuch gut verlaufen ist und mir 
die Tipps von Lorenz Müller weitergeholfen haben. Im Grossen und Ganzen bin ich sehr positiv eingestellt.

\subsubsection*{Was lief weniger gut}
Ausgenommen der Überschreitung der geschätzten Zeit für die Endpoints und das Dropdown lief heute alles gut.
Ich konnte die Fehler in meinem Konzept früh genug erkennen und habe nun mit der Zusammenfassung der Endpoints angemessen darauf reagiert.  

\subsubsection*{Meine Erkenntnisse von heute}
Auch wenn ein bereits erstelltes Konzept vorhanden ist, kann man diesem nie blind vertrauen. Man sollte es stets hinterfragen und prüfen
ob die Angaben darin immer noch korrekt sind. Evtl. waren noch nicht alle Informationen zum Zeitpunkt des Erstellens bekannt? Diese Aussage wird von meiner 
heutigen Entdeckung der Endpoints verifiziert.

\subsection*{Nächste Schritte}
Morgen werde ich zuerst die Endpoints dokumentieren. Danach muss das Dropdown fertiggestellt und die Turbo Streams mit den Partialänderungen abgeschlossen werden. 
Das Ziel ist es, morgen die Filterkriterien per Dropdown auf die Übersichtskomponente hinzufügen zu können. Die Partials an sich müssen noch nicht zu 100\% mit dem Mockup
übereinstimmen, dafür haben ich noch die weiteren eingeplanten eineinhalb Stunden am Freitag, aber das Hinzufügen sollte funktionieren.

\pagebreak
