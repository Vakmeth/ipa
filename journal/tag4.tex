\section{Tag 4: 07.03.2025}
\begin{table}[H]
    \begin{tabular}{|L{0.4\textwidth}|C{60pt}|C{60pt}|C{60pt}|}
        \hline
        \rowcolor{puzzleblue}\color{white}Tätigkeiten & \color{white}Beteiligte \color{white}Personen & \color{white}Aufwand Geplant (std) & \color{white}Aufwand Effektiv (std) \\
        \hline
         Analyse abschliessen & Marc Egli & 0 & 1 \\
        \hline
        Daily & Marc Egli & 0.25 & 0.25 \\
        \hline
        \textbf{Total} &  & Gesamtstunden soll & Gesamtstunden ist \\
        \hline
    \end{tabular}
    \caption{Tätigkeiten Tag 4}
\end{table}

\subsection*{Tagesablauf}
Heute morgen startete ich damit die funktionalen und nicht funktionalen Anforderungen zu dokumentieren. 
Danach startet um 09:00 das Daily. Im Daily präsentierte ich den Stand der IPA. Danach stellte ich diverse Fragen. Die
erste Frage war ob ich meinen Hauptexperten per Mail die die Frage zur Verwendung von Echtnamen in Diagrammen stellen könne. Dies
bestätigte mir Robin Steiner und Daniel Illi. Danach sprach ich mein gestriges Problem der Commit-Messages an. Robin Steiner und Daniel Illi
wiesen mich an, die Commits mit dem neuen Zeitstempel bestehen zu lassen und den Prozess des Fehlers bis hin zur Aufklärung im Daily
hier im Arbeitsjournal zu dokumentieren. Nach der Commit-Thematik habe ich eine Frage zum Kriterium A2 gestellt. Dieses besagt, dass alle
nicht gegebenen Informationen in der IPA identifiziert und dokumentiert werden müssen. Hierzu habe ich gefragt ob dieses Kriterium schon implizit durch das
Verfassen des Anhangs und der Dokumentation der Quelllen gegeben ist. Robin Steiner verneinte dies und wies mich an eine eigene Sektion in der Dokumentation
dafür zu erstellen.

Danach haben wir das Kriterium G5 diskutiert. Im Kriterium ist definiert, dass die Sicherheitsmassnahmen mit dem Team und den Stakeholdern
abgesprochen ist. Wir haben meine aufgeführten Risiken und Massnahmen dazu diskutiert, wobei mir Robin Steiner riet die Formulierung des Risikos von
Verwendung von Libraries mit Schwachstellen, neu zu formulieren. Ausserdem riet mit Daniel Illi für Risiken, welche die Berechtigungen des Benutzers betreffen,
Feature-Tests als Massnahme zu definieren. Als letzte fragte ich im Daily ob ich Anpassungen am Mockup machen dürfe. Der Fehler im Mockup ist mir heute morgen aufgefallen, 
als ich die funktionalen Anforderungen dokumentieret habe und das Mockup vor mir hatte. Es müsste ein Löschen-Button neben den Bearbeiten-Button in der Filterung hinzugefüt werden,
so dass der Benutzer auch Filterkriterien aus der Filterung entfernen kann. Ich präsentierte die Mockup-Änderung im Daily und bekam von meinen verantwortlichen Fachkräften
die Bestätigung um die Anpassung zu machen. 

Nachdem ich das Daily abgeschlossen hatte, schrieb ich den ersten Teil des Arbeitsjournals, da wir im Daily sehr viel besprochen hatten und ich
alle aufgekommenen Fragen und Anmerkungen zeitnah festhalten wollte.

\subsection*{Hilfestellungen}
\begin{itemize}
    \item Person: Hilfestellung
\end{itemize}

\subsection*{Reflexion}

\subsubsection*{Was lief gut}

\subsubsection*{Was lief weniger gut}

\subsubsection*{Meine Erkenntnisse von heute}

\subsection*{Nächste Schritte}

\pagebreak
