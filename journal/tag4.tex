\section{Tag 4: 07.03.2025}
\begin{table}[H]
    \begin{tabular}{|L{0.4\textwidth}|C{60pt}|C{60pt}|C{60pt}|}
        \hline
        \rowcolor{puzzleblue}\color{white}Tätigkeiten & \color{white}Beteiligte \color{white}Personen & \color{white}Aufwand Geplant (std) & \color{white}Aufwand Effektiv (std) \\
        \hline
        Analyse abschliessen & Marc Egli & 0 & 1 \\
        \hline
        Daily & Marc Egli, Daniel Illi, Robin Steiner & 0.25 & 0.25 \\
        \hline
        Anwendungskonzept ausarbeiten & Marc Egli & 1 & 1 \\
        \hline
        Systemkonzept verfassen & Marc Egli & 3 & 3 \\
        \hline
        Sicherheitskonzept verfassen & Marc Egli & 1 & 0.75 \\
        \hline
        Fehlerbehandlungskonzept erstellen & Marc Egli & 1 & 0.25 \\
        \hline
        Testkonzept erstellen & Marc Egli & 2 & 2 \\
        \hline
        Arbeitsjournal schreiben & Marc Egli & 0.25 & 0.25 \\
        \hline
        \textbf{Total} &  & 8.5 & 8.5 \\
    \hline
    \end{tabular}
    \caption{Tätigkeiten Tag 4}
\end{table}

\subsection*{Tagesablauf}
Heute morgen startete ich damit die funktionalen und nicht funktionalen Anforderungen zu dokumentieren. 
Danach startet um 09:00 das Daily. Im Daily präsentierte ich den Stand der IPA. Danach stellte ich diverse Fragen. Die
erste Frage war ob ich meinen Hauptexperten per Mail die die Frage zur Verwendung von Echtnamen in Diagrammen stellen könne. Dies
bestätigte mir Robin Steiner und Daniel Illi. Danach sprach ich mein gestriges Problem der Commit-Messages an. Robin Steiner und Daniel Illi
wiesen mich an, die Commits mit dem neuen Zeitstempel bestehen zu lassen und den Prozess des Fehlers bis hin zur Aufklärung im Daily
hier im Arbeitsjournal zu dokumentieren. Nach der Commit-Thematik habe ich eine Frage zum Kriterium A2 gestellt. Dieses besagt, dass alle
nicht gegebenen Informationen in der IPA identifiziert und dokumentiert werden müssen. Hierzu habe ich gefragt ob dieses Kriterium schon implizit durch das
Verfassen des Anhangs und der Dokumentation der Quelllen gegeben ist. Robin Steiner verneinte dies und wies mich an eine eigene Sektion in der Dokumentation
dafür zu erstellen.

Danach haben wir das Kriterium G5 diskutiert. Im Kriterium ist definiert, dass die Sicherheitsmassnahmen mit dem Team und den Stakeholdern
abgesprochen ist. Wir haben meine aufgeführten Risiken und Massnahmen dazu diskutiert, wobei mir Robin Steiner riet die Formulierung des Risikos von
Verwendung von Libraries mit Schwachstellen, neu zu formulieren. Ausserdem riet mit Daniel Illi für Risiken, welche die Berechtigungen des Benutzers betreffen,
Feature-Tests als Massnahme zu definieren. Als letzte fragte ich im Daily ob ich Anpassungen am Mockup machen dürfe. Der Fehler im Mockup ist mir heute morgen aufgefallen, 
als ich die funktionalen Anforderungen dokumentieret habe und das Mockup vor mir hatte. Es müsste ein Löschen-Button neben den Bearbeiten-Button in der Filterung hinzugefüt werden,
so dass der Benutzer auch Filterkriterien aus der Filterung entfernen kann. Ich präsentierte die Mockup-Änderung im Daily und bekam von meinen verantwortlichen Fachkräften
die Bestätigung um die Anpassung zu machen. 

Nachdem ich das Daily abgeschlossen hatte, schrieb ich den ersten Teil des Arbeitsjournals, da wir im Daily sehr viel besprochen hatten und ich
alle aufgekommenen Fragen und Anmerkungen zeitnah festhalten wollte.

Nachde ich das Daily abgeschlossen hatte, begann ich den grossen Zeitblock des Entwurfs auf kleiner Tickets umzulagern und diese in meinem Github Board aufzunehmen.
Dies tat ich da mich mein Hauptexperte daraufhin wies, kleiner Zeitblöcke einzuplanen. Direkt nachdem ich die Tickets erstellt hatte, begann ich mit dem Anwendungskonzept.
Ich konnte dieses zügig abschliessen un mich dann unmittelbar dem Systemkonzept widmen. Hier stellte ich während der Verfassung meiner Lösungsvarianten ein Probelm in der Aufgabenstellung
meiner IPA fest: Laut Aufgabenstellung ist es mir untersagt Anpassungen am Backend oder an den Endpoints zu machen. Genauso sei es Verboten, das Datenformat der Endpoints zu ändern. Allerdings
werde diese beiden Punkte umgehen müssen, denn:

\begin{itemize}
    \item A: Mein Feature ist auf das Datenformat ``Turbostreams'' ausgelegt, ich werde hier also zwingend Änderungen am Datenformat machen müssen
    \item B: Es müssen separate Endpoints für die verschiedenen Partials in der View angelegt werden, mit den bestehenden Endpoints ist das Feature nicht umsetzbar, da ich die einzelnen Filterkriterien nicht modular auswechseln kann
\end{itemize}

Ich werde diese Feststellung am Montag im Daily mit meinen verantwortlichen Fachkräften besprechen und mit ihnen die nächsten Massnahmen für dieses Problem festlegen.

Nachdem ich das Systemkonzept abgeschlossen hatte, habe ich als nächstes realisiert, dass ich noch nicht erfasste Schnittstellen in der Sicherheitsanalyse nachführen muss.
Durch die Analyse und den Entwurf meiner Lösung bemerkte ich dass ich dort noch nicht alle Schnittstellen erfasst habe. Dieses Problem habe ich auf meine Pendenzenliste geschrieben
und werde es Dienstagmorgen beheben.

Zuletzt begann ich mit dem Testkonzept. Obwohl ich bis dort gut in der Zeit war, reichte es mir nicht ganz fertig, weswegen ich den Restaufwand nächsten Dienstag erledigen werde.
Nebenbei schrieb ich Lorenz Müller eine Mail zur Nachfrage, ob ich Echtnamen in Diagrammen verwenden dürfe.

\subsection*{Hilfestellungen}
\begin{itemize}
    \item Robin Steiner und Daniel Illi: Nachfrage ob Hauptexperte per Mail kontaktiert werden darf
    \item Robin Steiner und Daniel Illi: Nachfrage Vorgehensweise Git-Commits
    \item Robin Steiner und Daniel Illi: Frage zu Kriterium A2, ob Anhang bereits als Nachweis zählt
    \item Robin Steiner und Daniel Illi: Diskussion der Sicherheitsrisiken und Massnahmen und somit Erfüllung von Punkt 4 in Kriterium G5.
    \item Robin Steiner und Daniel Illi: Nachfrage zur Formulierung des Sicherheitsrisikos zu Libraries
    \item Robin Steiner und Daniel Illi: Frage ob Mockup nachbearbeitet werden darf
\end{itemize}

\subsection*{Reflexion}

\subsubsection*{Was lief gut}
Persönlich war für mich ein Erfolg, dass ich die Probleme in der Aufgabenstellung und meinen eigenen Fehler bezüglich der Dokumentation der Schnittstellen
bei den Sicherheitsrisiken frühzeitig erkennen konnte. Dementsprechend kann ich auf diese Fehler noch vor der Umsetzung reagieren und mit einem sauberen Entwurf in diesen Sprint
starten. Ausserdem hatte ich das Gefühl, das ich heute meine Aufgaben besser organsiert hatte. Ich habe viel mehr mit Notizen für Anmerkungen von Robin und Daniel gearbeitet,
was es mir ermöglichte, auch das Arbeitsjournal detailtreuer zu schreiben.

\subsubsection*{Was lief weniger gut}
Heute war meine Konzentration verglichen zu gestern überhaupt nicht vorhanden. Ich hatte nur zum Teil das Gefühl das ich richtig in den 
Arbeitsfluss komme. Grund dafür könnte sein das heute Freitag ist und ich deswegen abgelenkter war. Deswegen konnte ich auch das Tagesziel von heute nicht erreichen, was bedeutet
das ich das Testkonzept am Montagmorgen fertigstellen muss.

\subsubsection*{Meine Erkenntnisse von heute}
Wichtig ist es Unklarheiten und mögliche Konflikte so früh wie möglich zu identifizieren, damit entsprechend auf diese eingegangen werden kann.

\subsection*{Nächste Schritte}
Am Dienstagmorgen werde ich zuerst die Schnittstellen der Sicherheitsmassnahmen ergänzen. Danach werde ich das Testkonzept abschliessen und den 
Sprintabschluss machen. Nach dem Sprintabschluss folgt wie dokumentiert das Planning und die Einteilung er nächsten Aufgaben für den kommenden Sprint.
Ziel ist es am Dienstag mit der Umsetzung zu starten.

\pagebreak
