\part[Projektdokumentation]{Projektdokumentation
                  \begin{center}
                     \begin{minipage}[c]{10.7cm}
                      \small Hitobito: Neue Generation von Personen-Filtern \\
                      Autor: Marc Egli
                     \end{minipage}
                  \end{center}
                 }
\chapter{Einführung}

\chapter{Analyse}
\section{Ist-Zustand}
\subsection{Personenlisten}
\subsection{Abonnemente}

\section{Soll-Zustand}

\section{Bedürfniserhebung}

\section{Risikoanalyse}

\section{Anforderungen}
\subsection{Nicht funktionale Anforderungen}
\subsection{Funktionale Anforderungen}

\section{Abgrenzung}

\section{Benötigter Rahmen}
\subsection{Fehlende Informationen}

\section{Persönliche Vorgehensziele}

\chapter{Entwurf}
\section{Anwendungskonzept}
\subsection{Anwendungsdiagram}
\subsection{Anwendungsfälle}

\section{Systemkonzept}
\subsection{Betroffene Services}
\subsection{Status quo}
\subsection{Lösungsvarianten}
\subsection{Variantenentscheid}

\section{Sicherheitskonzept}
\subsection{SQL-Injection}
\subsection{Cross-Site Scripting}
\subsection{URL Interpretation}
\subsection{Kommunikation HTTP/S}

\section{Fehlerbehandlungskonzept}
\subsection{Nutzereingabe}
\subsection{Laufzeitfehler}

\section{Testsetup}

\section{Testkonzept}
\subsection{Testinfrastruktur}
\subsection{Fehlerklassen}
\subsection{Manuelle Tests}

\chapter{Ausführung}
\section{Einsatz von KI-Modellen}
\section{Gems}
\subsection{can-can-can}
\subsection{dry-crud}

\chapter{Einführung}
\section{Instruktion}

\section{Unvorhergesehene Änderungen}
\subsection{application.rb}
\subsection{\_list.html.haml}

\chapter{Sprintabschlüsse}

\section{Abschluss Sprint Initialisierung}
\subsection{Backlog}

\section{Abschluss Sprint Umsetzung}
\subsection{Backlog}

\section{Abschluss Sprint Finalisierung}
\subsection{Backlog}


