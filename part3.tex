\part[Anhang und Verzeichnise]{Anhänge und Verzeichnise
                  \begin{center}
                     \begin{minipage}[c]{10.7cm}
                        \small Hitobito: Neue Generation von Personen-Filtern \\
                        Autor: Marc Egli
                     \end{minipage}
                  \end{center}
                 }

\chapter{Verzeichnise}

\section{Code}


\listoftables

\listoffigures

\renewcommand\bibname{Quellenverzeichnis}
\begin{thebibliography}{9}
    \bibitem[Verbindung zwischen repositories verstehen]{Github Docs} \url{https://docs.github.com/en/repositories/viewing-activity-and-data-for-your-repository/understanding-connections-between-repositories}, (04.03.2025)
    \bibitem[Issue Templates konfigurieren]{Github Docs} \url{https://docs.github.com/en/communities/using-templates-to-encourage-useful-issues-and-pull-requests/configuring-issue-templates-for-your-repository}, (04.03.2025)
    \bibitem[Translating]{Leo} \url{https://dict.leo.org/german-english}, (04.03.2025)
    \bibitem[Icon made by Freeplk from http://www.flaticon.com/]{Icon} \url{https://www.flaticon.com/free-icon/user_1077114?term=person&page=1&position=1&origin=search&related_id=1077114}, (04.03.2025)
    \bibitem[Product Owner Definition]{Agile Scrum Group} \url{https://agilescrumgroup.de/product-owner-aufgaben/}, (04.03.2025)
    \bibitem[Scrum Master Definition]{Agile Scrum Group} \url{https://agilescrumgroup.de/scrum-master-aufgaben/}, (04.03.2025)
    \bibitem[Entwickler Definition]{Agile Scrum Group} \url{https://scrumguide.de/entwickler/}, (04.03.2025)
    \bibitem[Bedürfniserhebung - Aufbau und Ablauf]{Easy-Feedback} \url{https://easy--feedback-de.translate.goog/umfrage-beispiele/bedarfsanalyse-fragebogen-vorlage/bedarfsanalyse-aufbau-ablauf-schritte/?_x_tr_sl=de&_x_tr_tl=en&_x_tr_hl=en&_x_tr_pto=sc}, (06.03.2025)
    \bibitem[Bedürfniserhebung - Interviews]{Kompass Digitale Kultur} \url{https://kreativ.mfg.de/digitale-kultur/kompass-digitale-kultur/prozess/nutzerinnen-gruppe/bedarfsanalyse-interviews/}, (06.03.2025)
    \bibitem[Ist-Situation Hitobito Bilder]{Hitobito Demoumgebung} \url{https://demo.hitobito.com}, (06.03.2025)
    \bibitem[Hochkommas in Latex]{Thinkscience} \url{https://thinkscience.co.jp/en/downloads/ThinkSCIENCE-LaTeX-habits-to-avoid.pdf}, (06.03.2025)
    \bibitem[Git Commit Messages bearbeiten]{Github Docs} \url{https://docs.github.com/en/pull-requests/committing-changes-to-your-project/creating-and-editing-commits/changing-a-commit-message}, (06.03.2025)
    \bibitem[Codeinput in Latex]{Overleaf} \url{https://www.overleaf.com/learn/latex/Code_listing}, (07.03.2025)
    \bibitem[XSS Attacks in Rails]{StackHawk} \url{https://www.stackhawk.com/blog/rails-xss-examples-and-prevention/}, (07.03.2025)
    \bibitem[Spezialzeichen in Latex]{Wikibooks} \url{https://en.wikibooks.org/wiki/LaTeX/Special_Characters}, (07.03.2025)
    \bibitem[Action in view auslesen]{StackOverflow} \url{https://stackoverflow.com/questions/8053312/rails-how-to-determine-controller-action-in-view}, (11.03.2025)
    \bibitem[User Story eines anderen Repositories referenzieren]{StackOverflow} \url{https://stackoverflow.com/questions/60268714/github-how-to-reference-an-issue-in-a-commit-from-a-different-repository}, (11.03.2025)
    \bibitem[Before Action in Controller]{StackOverflow} \url{https://stackoverflow.com/questions/45615591/how-does-only-at-before-action-work-in-rails}, (11.03.2025)
    \bibitem[Mapping in Rails]{Woman on Rails} \url{https://womanonrails.com/one-line-map-ruby}, (11.03.2025)
    \bibitem[HTML to HAML Converter]{AWSM-Tools} \url{https://awsm-tools.com/html-to-haml}, (11.03.2025)
    \bibitem[For-Loops in Rails Views]{StackOverflow} \url{https://stackoverflow.com/questions/13166847/rails-for-loop-view}, (11.03.2025)
    \bibitem[API Endpoints in Rails]{Dev.to - vladhilko} \url{https://dev.to/vladhilko/introduction-to-rails-api-how-to-create-your-first-endpoint-in-less-than-a-minute-4l66}, (12.03.2025)
    \bibitem[Variable in views]{StackOverflow} \url{https://stackoverflow.com/questions/15171850/how-to-create-and-use-variable-in-views-template-in-rails}, (13.03.2025)
    \bibitem[Rendering of Turbostreams]{Hotwire Discussion} \url{https://discuss.hotwired.dev/t/turbostream-cant-append-new-records-to-my-index-view/3569}, (13.03.2025)
    \bibitem[Delete Request für Turbostreams]{Hotwire Discussion} \url{https://stackoverflow.com/questions/75656122/how-to-have-a-delete-link-respond-to-turbo-stream-and-html-in-rails-7}, (13.03.2025)
    \bibitem[Template für JavaScript in Latex]{StackExchange} \url{https://tex.stackexchange.com/questions/89574/language-option-supported-in-listings}, (13.03.2025)
    
\end{thebibliography}
\addcontentsline{toc}{subsection}{Quellenverzeichnis}

\chapter{Verwendete Abkürzungen}

\begin{table}[H]
    \rowcolors{2}{puzzleblue!30}{white}
    \begin{tabular}{|L{0.3\textwidth}|L{0.6\textwidth}|}
        \hline
        \rowcolor{puzzleblue} \textbf{\color{white}Abkürzung} & \textbf{\color{white}Bedeutung} \\[12pt]
        \hline
        UML & Unified Modeling Language \\
        \hline
    \end{tabular}
    \caption{Verwendete Abkürzungen}
\end{table}

\chapter{Glossar}

\begin{table}[H]
    \rowcolors{2}{puzzleblue!30}{white}
    \begin{tabular}{|L{0.3\textwidth}|L{0.6\textwidth}|}
        \hline
        \rowcolor{puzzleblue} \textbf{\color{white}Bezeichnung} & \textbf{\color{white}Bedeutung} \\[12pt]
        \hline
        Hitobito & Community Management Tool \\
        \hline
    \end{tabular}
    \caption{Glossar}
\end{table}

\chapter{Anhänge}

\section{Git Commit Message Convention}
\label{sec:gitconv}
\begin{figure}[h]
    \centering
    \fbox{\includegraphics[width=1\textwidth,]{git_commit_conventions.png}}
    \caption{Puzzle ITC Git commit conventions}
\end{figure}

\section{Sitzungsprotokolle}
\subsection{Sitzung 1}

\begin{table}[H]
    \begin{tabular}{|L{0.3\textwidth}|L{0.7\textwidth}|}
        \hline
        \textbf{Datum} & 13.03.2025 \\
        \hline
        \textbf{Anwesende Personen} &
        \begin{itemize}[itemsep=0.5pt, topsep=0pt]
            \item Lorenz Müller
            \item Robin Steiner
            \item Daniel Illi
            \item Marc Egli
        \end{itemize} \\ 
        \hline
        \multicolumn{2}{|l|}{\textbf{Besprechungs Punkte}} \\
        \hline
        \textbf{Vorstellung} & Kurze Vorstellungsrunde \\
        \hline
        \textbf{Rollenverteilung} & Die Rollenverteilung während der IPA wurde nochmals kurz erläutert. \\
        \hline
        \textbf{Arbeitsplatz und Material} & Marc Egli bestätigte das alle nötigen Materialien vorhanden sind, er an keinem anderen Projekt arbeitet und er sich an seinem Arbeitsplatz konzentrieren kann.  \\
        \hline
        \textbf{Detailbeschrieb besprechen} & Der Detailbeschrieb wurde angeschaut, und Unklarheiten wurden geklärt. \\
        \hline
        \textbf{Bewertungskriterien Hinweise} & Erwähnte, dass der IPA-Kurzbericht nun im Teil 2 der IPA-Dokumentation hinterlegt werden muss. \\
        \hline
        \textbf{Bewertungskriterium KI-Einsatz} & Lorenz Müller wies darauf hin, dass das Kriterium C1: Einsatz von KI-Modellen falsch verstanden wurde. Es wurde festgelegt das 
        das Kriterium nach den festgelegten Punkten der verantwortlichen Fachkräfte bewertet wird.   \\
        \hline
        \textbf{Verwendung künstliche Intelligenz} & Lorenz Müller erklärte das die Verwendung von KI erlaubt ist, diese jedoch gekennzeichnet werden muss. \\
        \hline
        \textbf{Individuelle Bewertungskriterien} & Die Individuellen Bewertungskriterien wurden nochmals angeschaut und besprochen. \\
        \hline
    \end{tabular}
    \caption{Protokoll Sitzung 1.1}
\end{table}

\newpage

\begin{table}[H]
    \begin{tabular}{|L{0.3\textwidth}|L{0.7\textwidth}|}
        \hline
        \multicolumn{2}{|l|}{\textbf{Besprechungs Punkte}} \\
        \hline
        \textbf{Zeitplan besprechen} & Lorenz Müller wies daraufhin das der Aufbau der IPA mit drei Sprints nach Scrum nicht sinnvoll wäre. Um weiteren
        Zeitaufwand zu verhindern, wurde festgelegt, dass die Projektmethode und der Zeitplan wie vorberietet weitergeführt werden ohne dies mit einer Abwertung zu ahnden.
        Des weiteren wurde von Lorenz Müller angemerkt, dass stets Tätigkeiten im Zeitplan angegeben werden müssen. Wochenenden könnten ausserdem aus dem Zeitplan
        ausgeklammert werden. Die Planung soll ausserdem so umgeschrieben werden, dass 80 Stunden Aufwand für die IPA geschätzt wird. Tickets welche 8 Stunden einnehmen, sollen
        in weitere kleinere Tickets aufgeteilt werden. \\
        \hline
        \textbf{Arbeitsjournal besprechen} & Lorenz Müller erklärte, dass wichtige Erkenntnisse in der Reflexion vorkommen müssen, Anmerkungen zu der Zeit seien zu vernachlässigen,
        da diese in den geschätzten Stunden ersichtlich sind. \\
        \hline
        \textbf{Risikoanalyse besprechen} & Die Risikoanalyse wurde angeschaut. Lorenz Müller wies daraufhin, dass eine weiter Risikoanalyse für Projektrisiken
        angefertigt werden müsse. Diese Risikoanalyse soll im ersten Teil der IPA hinterlegt sein. \\
        \hline
        \textbf{Festlegung des zweiten Besuchtages} & Ein zweiter Besuchstag wurde am 12.03.2025 via Google Meet festgelegt, der Zugang für das Remote-Meeting wurde erteilt.\\
        \hline
        \textbf{Hinweise zu Tester und Lektor} & Lorenz Müller wies daraufhin, dass eine Person die IPA-Dokumentation gegenlesen und auf Rechtschreibefehler überprüfen darf. Diese Person 
        muss nicht erwähnt werden. Des weiteren darf ein Tester ausgewählt werden, welcher das Produkt prüft. Dieser muss angegeben werden, falls verwendet. \\
        \hline
    \end{tabular}
    \caption{Protokoll Sitzung 1.2}
\end{table}

\newpage

\begin{table}[H]
    \begin{tabular}{|L{0.3\textwidth}|L{0.7\textwidth}|}
        \hline
        \textbf{Datum} & 13.03.2025 \\
        \hline
        \textbf{Anwesende Personen} &
        \begin{itemize}[itemsep=0.5pt, topsep=0pt]
            \item Lorenz Müller
            \item Robin Steiner
            \item Daniel Illi
            \item Marc Egli
        \end{itemize} \\ 
        \hline
        \multicolumn{2}{|l|}{\textbf{Besprechungs Punkte}} \\
        \hline
        \textbf{Stand der IPA} & Marc Egli präsentiert den Stand der IPA \\
        \hline
        \textbf{Zeitplan} & Marc Egli präsentiert den Zeitplan \\
        \hline
        \textbf{Erläuterung Zeitverlust} & Marc Egli erklärt worauf der der Zeitverlust im ersten Sprint zurückzuführen ist.  \\
        \hline
        \textbf{Stand der Dokumentation überprüfen} & Marc Egli präsentiert die Dokumentation und stellt Fragen dazu, Lorenz Müller gibt Feedback und beantwortet Marc Eglis Fragen:
        \begin{itemize}
            \item Frage von Marc Egli: Muss das Datenschutzkonzept hinterlegt werden? 
            \item Antwort von Lorenz Müller: Nur falls es für diese Arbeit relevant ist. Hier gilt es Schwerpunkte zu setzen.
            \item Anmerkung von Lorenz Müller zu den Bildern: Bilder müssen auch bei Vergrösserung des Dokumentes noch zu lesen sein, hierdrauf achten.
            \item Frage von Marc Egli: Es wurde bereits ein Mockup für die zu implementierende Benutzerschnittstelle entworfen. Wie können die Anforderungen in diesem Fall lösungsneutral formuliert werden?
            \item Antwort von Lorenz Müller: Wenn das Mockup als Vorarbeit deklariert wurde kann hierdrauf Bezug genommen werden. Bei der lösungsneutralität geht es darum das entworfene Konzept zu hinterfragen und nicht einfach zu akzeptieren.
        \end{itemize}
        \\
        \hline
    \end{tabular}
    \caption{Protokoll Sitzung 2.1}
\end{table}

\newpage

\begin{table}[H]
    \begin{tabular}{|L{0.3\textwidth}|L{0.7\textwidth}|}
        \hline
        \textbf{Stand der Dokumentation überprüfen} & 
        \begin{itemize}
            \item Anmerkung von Lorenz Müller: Zu jeder Überschrift muss ein entsprechender Einleitungstext vorhanden sein, hierdrauf achten.
        \end{itemize} \\
        \textbf{Präsentation und Fachgespräch} & Lorenz Müller erklärt den Ablauf der Präsentation und des Fachgespräches
        \begin{itemize}
            \item Die Präsentation muss zwischen 15-20 Minuten dauern.
            \item Die Demo sollte zwischen 5-10 Minuten dauern. Keine Zeit verlieren, da die Zeit auf Kosten des Fachgespräches verloren geht.
            \item Im Fachgespräch werden Fragen zu sechs Fragenkomplexen kommen. Pro Fragenkomplex werden vier Unterfragen gestellt.
        \end{itemize}
        \\
        \hline
        \textbf{Terminvereinbarung} & Der Termin für Fachgespräch und Präsentation wird besprochen. Lorenz Müller merkt an, dass er eine Umfrage für den Termin aufschalten werde. \\
        \hline
        \textbf{Abschluss} & Lorenz Müller wies darauf hin, dass das Kriterium C1: Einsatz von KI-Modellen falsch verstanden wurde. Es wurde festgelegt das 
        das Kriterium nach den festgelegten Punkten der verantwortlichen Fachkräfte bewertet wird.   \\
        \hline
    \end{tabular}
    \caption{Protokoll Sitzung 2.2}
\end{table}

\section{Git commit convention}
\section{Security conventions}





