\part[Anhang und Verzeichnise]{Anhänge und Verzeichnise
                  \begin{center}
                     \begin{minipage}[c]{10.7cm}
                        \small Hitobito: Neue Generation von Personen-Filtern \\
                        Autor: Marc Egli
                     \end{minipage}
                  \end{center}
                 }

\chapter{Verzeichnise}

\section{Code}

\listoftables

\listoffigures

\renewcommand\bibname{Quellenverzeichnis}
\begin{thebibliography}{9}
    \bibitem[Agile Scrum Group - Product Owner]{Agile Scrum Group} \url{https://agilescrumgroup.de/product-owner-aufgaben/}, (14.01.2025)
\end{thebibliography}
\addcontentsline{toc}{subsection}{Quellenverzeichnis}

\chapter{Verwendete Abkürzungen}

\begin{table}[H]
    \rowcolors{2}{puzzleblue!30}{white}
    \begin{tabular}{|L{0.3\textwidth}|L{0.6\textwidth}|}
        \hline
        \rowcolor{puzzleblue} \textbf{\color{white}Abkürzung} & \textbf{\color{white}Bedeutung} \\[12pt]
        \hline
        UML & Unified Modeling Language \\
        \hline
    \end{tabular}
    \caption{Verwendete Abkürzungen}
\end{table}

\chapter{Glossar}

\begin{table}[H]
    \rowcolors{2}{puzzleblue!30}{white}
    \begin{tabular}{|L{0.3\textwidth}|L{0.6\textwidth}|}
        \hline
        \rowcolor{puzzleblue} \textbf{\color{white}Bezeichnung} & \textbf{\color{white}Bedeutung} \\[12pt]
        \hline
        Hitobito & Community Management Tool \\
        \hline
    \end{tabular}
    \caption{Glossar}
\end{table}

\chapter{Anhänge}

\section{Sitzungsprotokolle}
\section{Git commit convention}
\section{Security conventions}





