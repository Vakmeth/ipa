\part[Anhang und Verzeichnise]{Anhänge und Verzeichnise
                  \begin{center}
                     \begin{minipage}[c]{10.7cm}
                        \small Hitobito: Neue Generation von Personen-Filtern \\
                        Autor: Marc Egli
                     \end{minipage}
                  \end{center}
                 }

\chapter{Verzeichnise}

\section{Code}


\listoftables

\listoffigures

\renewcommand\bibname{Quellenverzeichnis}
\begin{thebibliography}{9}
    \bibitem[Github Docs - Understanding connections between repositories]{Github Docs} \url{https://docs.github.com/en/repositories/viewing-activity-and-data-for-your-repository/understanding-connections-between-repositories}, (04.03.2025)
    \bibitem[Github Docs - Configuring issue templates]{Github Docs} \url{https://docs.github.com/en/communities/using-templates-to-encourage-useful-issues-and-pull-requests/configuring-issue-templates-for-your-repository}, (04.03.2025)
    \bibitem[Leo - Translating]{Leo} \url{https://dict.leo.org/german-english}, (04.03.2025)
    \bibitem[Icon made by Freeplk from http://www.flaticon.com/]{Icon} \url{https://www.flaticon.com/free-icon/user_1077114?term=person&page=1&position=1&origin=search&related_id=1077114}, (04.03.2025)
    \bibitem[Agile Scrum Group - Product Owner]{Agile Scrum Group} \url{https://agilescrumgroup.de/product-owner-aufgaben/}, (04.03.2025)
    \bibitem[Agile Scrum Group - Scrum Master]{Agile Scrum Group} \url{https://agilescrumgroup.de/scrum-master-aufgaben/}, (04.03.2025)
    \bibitem[Agile Scrum Group - Entwickler]{Agile Scrum Group} \url{https://scrumguide.de/entwickler/}, (04.03.2025)
    \bibitem[Bedürfniserhebung - Aufbau und Ablauf]{Easy-Feedback} \url{https://easy--feedback-de.translate.goog/umfrage-beispiele/bedarfsanalyse-fragebogen-vorlage/bedarfsanalyse-aufbau-ablauf-schritte/?_x_tr_sl=de&_x_tr_tl=en&_x_tr_hl=en&_x_tr_pto=sc}, (06.03.2025)
    \bibitem[Bedürfniserhebung - Interviews]{Kompass Digitale Kultur} \url{https://kreativ.mfg.de/digitale-kultur/kompass-digitale-kultur/prozess/nutzerinnen-gruppe/bedarfsanalyse-interviews/}, (06.03.2025)
    \bibitem[Ist-Situation Hitobito Bilder]{Hitobito Demoumgebung} \url{https://demo.hitobito.com}, (06.03.2025)
    \bibitem[Hochkommas in Latex]{Thinkscience} \url{https://thinkscience.co.jp/en/downloads/ThinkSCIENCE-LaTeX-habits-to-avoid.pdf}, (06.03.2025)
    
\end{thebibliography}
\addcontentsline{toc}{subsection}{Quellenverzeichnis}

\chapter{Verwendete Abkürzungen}

\begin{table}[H]
    \rowcolors{2}{puzzleblue!30}{white}
    \begin{tabular}{|L{0.3\textwidth}|L{0.6\textwidth}|}
        \hline
        \rowcolor{puzzleblue} \textbf{\color{white}Abkürzung} & \textbf{\color{white}Bedeutung} \\[12pt]
        \hline
        UML & Unified Modeling Language \\
        \hline
    \end{tabular}
    \caption{Verwendete Abkürzungen}
\end{table}

\chapter{Glossar}

\begin{table}[H]
    \rowcolors{2}{puzzleblue!30}{white}
    \begin{tabular}{|L{0.3\textwidth}|L{0.6\textwidth}|}
        \hline
        \rowcolor{puzzleblue} \textbf{\color{white}Bezeichnung} & \textbf{\color{white}Bedeutung} \\[12pt]
        \hline
        Hitobito & Community Management Tool \\
        \hline
    \end{tabular}
    \caption{Glossar}
\end{table}

\chapter{Anhänge}

\section{Git Commit Message Convention}
\label{sec:gitconv}
\begin{figure}[h]
    \centering
    \fbox{\includegraphics[width=1\textwidth,]{git_commit_conventions.png}}
    \caption{Puzzle ITC Git commit conventions}
\end{figure}

\section{Daily-Protokolle}
\label{sec:dailyprot}

\section{Sitzungsprotokolle}

\section{Git commit convention}
\section{Security conventions}





