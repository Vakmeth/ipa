\part[Anhang und Verzeichnisse]{Anhänge und Verzeichnise
                  \begin{center}
                     \begin{minipage}[c]{10.7cm}
                        \small Hitobito: Neue Generation von Personen-Filtern \\
                        Autor: Marc Egli
                     \end{minipage}
                  \end{center}
                 }

\chapter{Verzeichnisse}

\listoftables

\listoffigures

\renewcommand\bibname{Quellenverzeichnis}
\begin{thebibliography}{9}
    \bibitem[Verbindung zwischen repositories verstehen]{Github Docs} \url{https://docs.github.com/en/repositories/viewing-activity-and-data-for-your-repository/understanding-connections-between-repositories}, (04.03.2025)
    \bibitem[Issue Templates konfigurieren]{Github Docs} \url{https://docs.github.com/en/communities/using-templates-to-encourage-useful-issues-and-pull-requests/configuring-issue-templates-for-your-repository}, (04.03.2025)
    \bibitem[Translating]{Leo} \url{https://dict.leo.org/german-english}, (04.03.2025)
    \bibitem[Icon made by Freeplk from http://www.flaticon.com/]{Icon} \url{https://www.flaticon.com/free-icon/user_1077114?term=person&page=1&position=1&origin=search&related_id=1077114}, (04.03.2025)
    \bibitem[Product Owner Definition]{Agile Scrum Group} \url{https://agilescrumgroup.de/product-owner-aufgaben/}, (04.03.2025)
    \bibitem[Scrum Master Definition]{Agile Scrum Group} \url{https://agilescrumgroup.de/scrum-master-aufgaben/}, (04.03.2025)
    \bibitem[Entwickler Definition]{Agile Scrum Group} \url{https://scrumguide.de/entwickler/}, (04.03.2025)
    \bibitem[Bedürfniserhebung - Aufbau und Ablauf]{Easy-Feedback} \url{https://easy--feedback-de.translate.goog/umfrage-beispiele/bedarfsanalyse-fragebogen-vorlage/bedarfsanalyse-aufbau-ablauf-schritte/?_x_tr_sl=de&_x_tr_tl=en&_x_tr_hl=en&_x_tr_pto=sc}, (06.03.2025)
    \bibitem[Bedürfniserhebung - Interviews]{Kompass Digitale Kultur} \url{https://kreativ.mfg.de/digitale-kultur/kompass-digitale-kultur/prozess/nutzerinnen-gruppe/bedarfsanalyse-interviews/}, (06.03.2025)
    \bibitem[Ist-Situation Hitobito Bilder]{Hitobito Demoumgebung} \url{https://demo.hitobito.com}, (06.03.2025)
    \bibitem[Hochkommas in Latex]{Thinkscience} \url{https://thinkscience.co.jp/en/downloads/ThinkSCIENCE-LaTeX-habits-to-avoid.pdf}, (06.03.2025)
    \bibitem[Git Commit Messages bearbeiten]{Github Docs} \url{https://docs.github.com/en/pull-requests/committing-changes-to-your-project/creating-and-editing-commits/changing-a-commit-message}, (06.03.2025)
    \bibitem[Codeinput in Latex]{Overleaf} \url{https://www.overleaf.com/learn/latex/Code_listing}, (07.03.2025)
    \bibitem[XSS Attacks in Rails]{StackHawk} \url{https://www.stackhawk.com/blog/rails-xss-examples-and-prevention/}, (07.03.2025)
    \bibitem[Spezialzeichen in Latex]{Wikibooks} \url{https://en.wikibooks.org/wiki/LaTeX/Special_Characters}, (07.03.2025)
    \bibitem[Action in view auslesen]{StackOverflow} \url{https://stackoverflow.com/questions/8053312/rails-how-to-determine-controller-action-in-view}, (11.03.2025)
    \bibitem[User Story eines anderen Repositories referenzieren]{StackOverflow} \url{https://stackoverflow.com/questions/60268714/github-how-to-reference-an-issue-in-a-commit-from-a-different-repository}, (11.03.2025)
    \bibitem[Before Action in Controller]{StackOverflow} \url{https://stackoverflow.com/questions/45615591/how-does-only-at-before-action-work-in-rails}, (11.03.2025)
    \bibitem[Mapping in Rails]{Woman on Rails} \url{https://womanonrails.com/one-line-map-ruby}, (11.03.2025)
    \bibitem[HTML to HAML Converter]{AWSM-Tools} \url{https://awsm-tools.com/html-to-haml}, (11.03.2025)
    \bibitem[For-Loops in Rails Views]{StackOverflow} \url{https://stackoverflow.com/questions/13166847/rails-for-loop-view}, (11.03.2025)
    \bibitem[API Endpoints in Rails]{Dev.to - vladhilko} \url{https://dev.to/vladhilko/introduction-to-rails-api-how-to-create-your-first-endpoint-in-less-than-a-minute-4l66}, (12.03.2025)
    \bibitem[Variable in views]{StackOverflow} \url{https://stackoverflow.com/questions/15171850/how-to-create-and-use-variable-in-views-template-in-rails}, (13.03.2025)
    \bibitem[Rendering of Turbostreams]{Hotwire Discussion} \url{https://discuss.hotwired.dev/t/turbostream-cant-append-new-records-to-my-index-view/3569}, (13.03.2025)
    \bibitem[Delete Request für Turbostreams]{Hotwire Discussion} \url{https://stackoverflow.com/questions/75656122/how-to-have-a-delete-link-respond-to-turbo-stream-and-html-in-rails-7}, (13.03.2025)
    \bibitem[Template für JavaScript in Latex]{StackExchange} \url{https://tex.stackexchange.com/questions/89574/language-option-supported-in-listings}, (13.03.2025)
    \bibitem[Tom-Select]{Tom-Select.js} \url{https://tom-select.js.org/examples/options/}, (14.03.2025)
    \bibitem[Übersetzung von Texten für die View]{Leo} \url{https://dict.leo.org/german-english/datum}, (14.03.2025)
    \bibitem[Turbostreams vs Turboframes]{Mix & Go} \url{https://youtu.be/vnDWsGtzOCc}, (19.03.2025)
\end{thebibliography}
\addcontentsline{toc}{subsection}{Quellenverzeichnis}

\chapter{Verwendete Abkürzungen}

\begin{table}[H]
    \rowcolors{2}{puzzleblue!30}{white}
    \begin{tabular}{|L{0.3\textwidth}|L{0.6\textwidth}|}
        \hline
        \rowcolor{puzzleblue} \textbf{\color{white}Abkürzung} & \textbf{\color{white}Bedeutung} \\[12pt]
        CSV & Comma seperated values \\
        \hline
        DoD & Definition of Done \\
        \hline
        HAML & HTML Abstraction Markup Language \\
        \hline
        HTML & HyperText Markup Language \\
        \hline
        HTTP & HyperText Transfer Protocol \\
        \hline
        HTTPS & HyperText Transfer Protocol Secure \\
        \hline
        IPA & Individuelle Praktische Arbeit \\
        \hline
        KI & Künstliche Intelligenz \\
        \hline
        ORM & Object-Relational Mapping \\
        \hline
        UI & User Interface \\
        \hline
        UML & Unified Modeling Language \\
        \hline
        USB & Universal Serial Bus \\
        \hline
        UX & User Experience \\
        \hline
        VF & Verantwortliche Fachkraft \\
        \hline
        XSS & Cross-site scripting \\
        \hline
    \end{tabular}
    \caption{Verwendete Abkürzungen}
\end{table}

\chapter{Glossar}

\begin{table}[H]
    \rowcolors{2}{puzzleblue!30}{white}
    \begin{tabular}{|L{0.3\textwidth}|L{0.6\textwidth}|}
        \hline
        \rowcolor{puzzleblue} \textbf{\color{white}Bezeichnung} & \textbf{\color{white}Bedeutung} \\[12pt]
        \hline
        Active Record & ORM in Rails \\
        \hline
        Backend & Serverseitiger Teil einer Applikation, welcher für den Benutzer nicht sichtbar ist \\
        \hline
        Core & Wagon welcher die Grundfunktionalität von Hitobito beinhaltet \\
        \hline 
        Div & Container für Element in der Benutzerschnittstelle \\
        \hline
        Formatter & Software welche die Dateien automatisch formatiert \\
        \hline
        Frontend & Teil der Applikation mit welcher der Benutzer direkt interagiert \\
        \hline
        Gem & Library als Rubycode welcher bestimmte Funktionalität zur Verfügung stellt \\
        \hline
        Hitobito & Community Management Tool \\
        \hline
        Hotwire & Frontend Framework für Ruby on Rails \\
        \hline
        Mockup & Entwurf einer Benutzerschnittstelle, dient der Orientierung während der Umsetzung \\
        \hline
        Partials & Benutzerschnittstelle welche wiederverwendet wird \\
        \hline
        Product Backlog & Ort an welchem die User Stories verwaltet werden \\
        \hline
        Scrum & Agile Projektvorgehensmethode \\
        \hline
        SSH-Key & Zugang zum SSH-Protokoll, vergeleichbar zu einem Benutzernamen mit Passwort \\
        \hline
        Stakeholder & Synonim für ``Kunde'' im Scrum Prozess \\
        \hline
        Transifex & Software in welcher Übersetzter Wörter übersetzten und diese dann als .yml Datei zur Verfügung gestellt werden \\
        \hline
        Turboframe & Container innerhalb einer Benutzerschnittstelle welcher sich auswechseln lässt \\
        \hline
        Turbostreams & Technologie um einen Teil der Benutzerschnittstelle zu aktualisieren \\
        \hline
        Wagon & Zusätzliche Funktionalität für jeweiligen Kunden. Kann an Core angehängt werden \\
        \hline
    \end{tabular}
    \caption{Glossar}
\end{table}

\chapter{Anhänge}
Der Anhang verwaltet Links und Dokumente welche im Rahmen der IPA erstellt oder 
verwendet wurden.

\section{Github Repositories}
Die folgenden zwei Links führen zum Repository der Dokumentation und des Codes.

Repository zur Dokumentation: \href{https://github.com/Vakmeth/ipa}{Dokumentation}

Repository zum Code: \href{https://github.com/Vakmeth/hitobito}{Code}

\section{Git commit Message Konvention}
\label{sec:gitconv}
\begin{figure}[h]
    \centering
    \fbox{\includegraphics[width=1\textwidth,]{git_commit_conventions.png}}
    \caption{Git commit Message Konvention}
\end{figure}

\section{Sitzungsprotokolle}
Die Sitzungsprotokolle dokumentieren den ersten und zweiten Besuchstag der 
Experten während der Durchführung der IPA. 

\subsection{Sitzung 1}

\begin{table}[H]
    \begin{tabular}{|L{0.3\textwidth}|L{0.7\textwidth}|}
        \hline
        \textbf{Datum} & 13.03.2025 \\
        \hline
        \textbf{Anwesende Personen} &
        \begin{itemize}[itemsep=0.5pt, topsep=0pt]
            \item Lorenz Müller
            \item Robin Steiner
            \item Daniel Illi
            \item Marc Egli
        \end{itemize} \\ 
        \hline
        \multicolumn{2}{|l|}{\textbf{Besprechungspunkte}} \\
        \hline
        \textbf{Vorstellung} & Kurze Vorstellungsrunde \\
        \hline
        \textbf{Rollenverteilung} & Die Rollenverteilung während der IPA wurde nochmals kurz erläutert. \\
        \hline
        \textbf{Arbeitsplatz und Material} & Marc Egli bestätigte, dass alle nötigen Materialien vorhanden sind, er an keinem anderen Projekt arbeitet und er sich an seinem Arbeitsplatz konzentrieren kann.  \\
        \hline
        \textbf{Detailbeschrieb besprechen} & Der Detailbeschrieb wurde angeschaut, Unklarheiten wurden geklärt. \\
        \hline
        \textbf{Bewertungskriterien Hinweise} & Lorenz Müller erwähnte, dass der IPA-Kurzbericht nun im Teil 2 der IPA-Dokumentation hinterlegt werden muss. \\
        \hline
        \textbf{Bewertungskriterium KI-Einsatz} & Lorenz Müller wies darauf hin, dass das Kriterium C1: Einsatz von KI-Modellen falsch verstanden wurde. Es wurde festgelegt das 
        das Kriterium nach den festgelegten Punkten der verantwortlichen Fachkräfte bewertet wird.   \\
        \hline
        \textbf{Verwendung künstliche Intelligenz} & Lorenz Müller erklärte das die Verwendung von KI erlaubt ist, diese jedoch gekennzeichnet werden muss. \\
        \hline
        \textbf{Individuelle Bewertungskriterien} & Die Individuellen Bewertungskriterien wurden nochmals angeschaut und besprochen. \\
        \hline
    \end{tabular}
    \caption{Protokoll Sitzung 1.1}
\end{table}

\newpage

\begin{table}[H]
    \begin{tabular}{|L{0.3\textwidth}|L{0.7\textwidth}|}
        \hline
        \multicolumn{2}{|l|}{\textbf{Besprechungs Punkte}} \\
        \hline
        \textbf{Zeitplan besprechen} & Lorenz Müller wies daraufhin, dass der Aufbau der IPA mit drei Sprints nach Scrum nicht sinnvoll wäre. Um weiteren
        Zeitaufwand zu verhindern, wurde festgelegt, dass die Projektmethode und der Zeitplan wie vorbereitet weitergeführt werden, ohne dies mit einer Abwertung zu ahnden.
        Des Weiteren wurde von Lorenz Müller angemerkt, dass stets Tätigkeiten im Zeitplan angegeben werden müssen. Wochenenden könnten ausserdem aus dem Zeitplan
        ausgeklammert werden. Die Planung soll ausserdem so umgeschrieben werden, dass 80 Stunden Aufwand für die IPA geschätzt wird. Tickets welche 8 Stunden einnehmen, sollen
        in weitere kleinere Tickets aufgeteilt werden. \\
        \hline
        \textbf{Arbeitsjournal besprechen} & Lorenz Müller erklärte, dass wichtige Erkenntnisse in der Reflexion vorkommen müssen, Anmerkungen zu der Zeit seien zu vernachlässigen,
        da diese in den geschätzten Stunden ersichtlich sind. \\
        \hline
        \textbf{Risikoanalyse besprechen} & Die Risikoanalyse wurde angeschaut. Lorenz Müller wies daraufhin, dass eine weiter Risikoanalyse für Projektrisiken
        angefertigt werden müsse. Diese Risikoanalyse soll im ersten Teil der IPA hinterlegt sein. \\
        \hline
        \textbf{Festlegung des zweiten Besuchtages} & Ein zweiter Besuchstag wurde am 12.03.2025 via Google Meet festgelegt, der Zugang für das Remote-Meeting wurde erteilt.\\
        \hline
        \textbf{Hinweise zu Tester und Rechtschreibeprüfer} & Lorenz Müller wies daraufhin, dass eine Person die IPA-Dokumentation gegenlesen und auf Rechtschreibefehler überprüfen darf. Diese Person 
        muss nicht erwähnt werden. Des Weiteren darf ein Tester ausgewählt werden, welcher das Produkt prüft. Dieser muss angegeben werden, falls verwendet. \\
        \hline
    \end{tabular}
    \caption{Protokoll Sitzung 1.2}
\end{table}

\newpage

\subsection{Sitzung 2}

\begin{table}[H]
    \begin{tabular}{|L{0.3\textwidth}|L{0.7\textwidth}|}
        \hline
        \textbf{Datum} & 13.03.2025 \\
        \hline
        \textbf{Anwesende Personen} &
        \begin{itemize}[itemsep=0.5pt, topsep=0pt]
            \item Lorenz Müller
            \item Robin Steiner
            \item Daniel Illi
            \item Marc Egli
        \end{itemize} \\ 
        \hline
        \multicolumn{2}{|l|}{\textbf{Besprechungs Punkte}} \\
        \hline
        \textbf{Stand der IPA} & Marc Egli präsentiert den Stand der IPA \\
        \hline
        \textbf{Zeitplan} & Marc Egli präsentiert den Zeitplan \\
        \hline
        \textbf{Erläuterung Zeitverlust} & Marc Egli erklärt, worauf der Zeitverlust im ersten Sprint zurückzuführen ist.  \\
        \hline
        \textbf{Stand der Dokumentation überprüfen} & Marc Egli präsentiert die Dokumentation und stellt Fragen dazu, Lorenz Müller gibt Feedback und beantwortet Marc Egli's Fragen:
        \begin{itemize}
            \item Frage von Marc Egli: Muss das Datenschutzkonzept hinterlegt werden? 
            \item Antwort von Lorenz Müller: Nur falls es für diese Arbeit relevant ist. Hier gilt es Schwerpunkte zu setzen.
            \item Anmerkung von Lorenz Müller zu den Bildern: Bilder müssen auch bei Vergrösserung des Dokumentes noch zu lesen sein, darauf achten.
            \item Frage von Marc Egli: Es wurde bereits ein Mockup für die zu implementierende Benutzerschnittstelle entworfen. Wie können die Anforderungen in diesem Fall lösungsneutral formuliert werden?
            \item Antwort von Lorenz Müller: Wenn das Mockup als Vorarbeit deklariert wurde, kann darauf Bezug genommen werden. Bei der Lösungsneutralität geht es darum das entworfene Konzept zu hinterfragen und nicht einfach zu akzeptieren.
        \end{itemize}
        \\
        \hline
    \end{tabular}
    \caption{Protokoll Sitzung 2.1}
\end{table}

\newpage

\begin{table}[H]
    \begin{tabular}{|L{0.3\textwidth}|L{0.7\textwidth}|}
        \hline
        \textbf{Stand der Dokumentation überprüfen} & 
        \begin{itemize}
            \item Anmerkung von Lorenz Müller: Zu jeder Überschrift muss ein entsprechender Einleitungstext vorhanden sein, darauf achten.
        \end{itemize} \\
        \textbf{Präsentation und Fachgespräch} & Lorenz Müller erklärt den Ablauf der Präsentation und des Fachgesprächs
        \begin{itemize}
            \item Die Präsentation muss zwischen 15-20 Minuten dauern.
            \item Die Demo sollte zwischen 5-10 Minuten dauern. Keine Zeit verlieren, da die Zeit auf Kosten des Fachgesprächs verloren geht.
            \item Im Fachgespräch werden Fragen zu sechs Fragenkomplexen kommen. Pro Fragenkomplex werden vier Unterfragen gestellt.
        \end{itemize}
        \\
        \hline
        \textbf{Terminvereinbarung} & Der Termin für Fachgespräch und Präsentation wird besprochen. Lorenz Müller merkt an, dass er eine Umfrage für den Termin aufschalten werde. \\
        \hline
        \textbf{Abschluss} & Teilnehmer verabschieden sich.   \\
        \hline
    \end{tabular}
    \caption{Protokoll Sitzung 2.2}
\end{table}

\newpage

\section{Code}

\subsection{people\_filter\_ability.rb}
\begin{lstlisting}[language=Ruby]
class PeopleFilterAbility < AbilityDsl::Base
   on(::PeopleFilter) do
     permission(:contact_data).may(:new).all
     permission(:group_read).may(:new).in_same_group
+    permission(:group_read).may(:filter_criterion).in_same_group
     permission(:group_and_below_read).may(:new).in_same_group_or_below
+    permission(:group_and_below_read).may(:filter_criterion).in_same_group_or_below
     permission(:layer_read).may(:new).in_same_layer
+    permission(:layer_read).may(:filter_criterion).in_same_layer
     permission(:layer_full).may(:create, :destroy, :edit, :update).in_same_layer
     permission(:layer_and_below_read).may(:new).in_same_layer_or_below
+    permission(:layer_and_below_read).may(:filter_criterion).in_same_layer_or_below
     permission(:layer_and_below_full).may(:create, :destroy, :edit, :update).in_same_layer_or_below
   end
end
\end{lstlisting}

\subsection{people\_readables.rb}
\begin{lstlisting}[language=Ruby]
class PersonReadables < GroupBasedReadables
    attr_reader :group
  
 -  def initialize(user, group = nil)
 +  def initialize(user, group = nil, roles_join = nil)
      super(user)
  
 +    @roles_join = roles_join || {roles: :group}
      @group = group
  
      if @group.nil?
    end

    def accessible_people
      if user.root?
        Person.only_public_data.then do |scope|
 -        group ? scope.joins(roles: :group).distinct : scope
 +        group ? scope.joins(@roles_join).distinct : scope
        end
      else
        scope = Person.only_public_data.where(accessible_conditions.to_a).distinct
        if has_group_based_conditions?
          # Only add these joins when really necessary, because they are extremely expensive to
          # compute when there are a lot of people and roles
 -        scope = scope.joins(roles: :group).where(groups: {deleted_at: nil})
 +        scope = scope.joins(@roles_join).where(groups: {deleted_at: nil})
        end
        scope
    end
end    
\end{lstlisting}

\newpage

\subsection{people\_filters\_controller.rb}
\begin{lstlisting}[language=Ruby]
class PeopleFiltersController < CrudController
    skip_authorize_resource only: [:create]
  
 +  before_action :set_filter_criteria, except: [:destroy]
 +
 +  def filter_criterion
 +    compose_role_lists
 +    possible_tags
 +    @filter_criterion = params[:filter_criterion]
 +    if @filter_criteria.include?(@filter_criterion.to_sym)
 +      respond_to do |format|
 +        if request.method == "GET"
 +          format.turbo_stream { render "create", status: :ok }
 +        end
 +        if request.method == "POST"
 +          format.turbo_stream { render "delete" }
 +        end
 +      end
 +    end
 +  end
 +
 +  def set_filter_criteria
 +    @filter_criteria = [:tag, :role, :qualification, :attributes]
 +  end

    def compose_role_lists
        @role_types = Role::TypeList.new(group.class)
        @qualification_kinds = QualificationKind.list.without_deleted
+           .map { |qualification|
+           [qualification.label, qualification.id, qualification.id]
+       }
+       @roles = Role.all.map { |role| [role.type, role.id, role.id] }
+       @kinds = Person::Filter::Role::KINDS.each_with_index
+           .map { |kind, index|
+           [t("people_filters.form.filters_role_kind.#{kind}"),
+           t("people_filters.form.filters_role_kind.#{kind}"),
+           index + 1]
+       }
+       @validities = [
+           [t("people_filters.qualification.validity_label.active"), "active", 1],
+           [t("people_filters.qualification.validity_label.reactivateable"), "reactivateable", 2],
+           [t("people_filters.qualification.validity_label.not_active_but_reactivateable"), "not_active_but_reactivateable", 3],
+           [t("people_filters.qualification.validity_label.not_active"), "not_active", 4],
+           [t("people_filters.qualification.validity_label.all"), "all", 5],
+           [t("people_filters.qualification.validity_label.none"), "none", 6],
+           [t("people_filters.qualification.validity_label.only_expired"), "only_expired", 7]
+       ]
    end
end    
\end{lstlisting}

\newpage

\subsection{people\_filter\_helper.rb}
\begin{lstlisting}[language=Ruby]
module PeopleFilterHelper
       content << content_tag(:div, class: "flex-none") do
-        select_tag("filters[attributes][#{time}][key]",
-          options_from_collection_for_select(people_filter_attributes_for_select, :last, :first, key),
+        select(:filters, "attributes[#{time}][key]",
+          people_filter_attributes_for_select,
+          {selected: key},
           html_options.merge(disabled: true,
             class: 'attribute_key_dropdown form-select
                                                   form-select-sm'))
       end
 
       content << content_tag(:div, class: "flex-none") do
-        select_tag("filters[attributes][#{time}][constraint]",
-          options_from_collection_for_select(filters, :last, :first, constraint),
+        select(:filters, "attributes[#{time}][constraint]",
+          filters,
+          {selected: constraint},
           html_options.merge(class:
-                             'attribute_constraint_dropdown
-                                         ms-3 form-select form-select-sm'))
+                               'attribute_constraint_dropdown
+                                         ms-3 form-select form-select-sm', name: "filters[attributes][#{time}][constraint]"))
       end
end
\end{lstlisting}

\newpage

\subsection{filter\_dropdown\_controller.js}
\begin{lstlisting}[language=JavaScript]
+// Copyright (c) 2023, Schweizer Alpen-Club. This file is part of
+// hitobito_sac_cas and licensed under the Affero General Public License version 3
+// or later. See the COPYING file at the top-level directory or at
+// https://github.com/hitobito/hitobito
+
+import { Controller } from "@hotwired/stimulus";
+
+export default class extends Controller {
+   static targets = ["menu"];
+
+  connect() {
+    this.observer = new MutationObserver(this.checkIfEmpty.bind(this));
+    if (this.hasMenuTarget) {
+      this.observer.observe(this.menuTarget, { childList: true });
+    }
+  }
+
+  disconnect() {
+    if (this.observer) {
+      this.observer.disconnect();
+    }
+  }
+
+  checkIfEmpty() {
+    if (this.menuTarget.children.length === 0) {
+      this.element.style.visibility = "hidden";
+    } else {
+      this.element.style.visibility = "visible";
+    }
+  }
+}
\end{lstlisting}

\newpage

\subsection{form\_field\_toggle\_controller.js}
\begin{lstlisting}[language=JavaScript]
export default class extends Controller {
        static targets = ["toggle"];
      
        toggle(event) {
-           if(event.target.tagName === "SELECT") {
-               const selected = 
                    event.target.options[event.target.options.selectedIndex];
+               const selected = 
                    event.target.options[event.target.options.selectedIndex];
      
-           if (selected.dataset.visibility === "true") {
-               this.toggleTarget.classList.remove("hidden");
-           } else {
-               this.toggleTarget.classList.add("hidden");
-           }
-           } else {
+               if (selected.dataset.visibility === "true") {
                    this.toggleTarget.classList.remove("hidden");
+               } else {
+                   this.toggleTarget.classList.add("hidden");
                }
            }
-
-           untoggle() {
-               this.toggleTarget.classList.add("hidden");
-           }
    }    
}    
\end{lstlisting}

\newpage

\subsection{form\_select\_controller.js}
\begin{lstlisting}[language=JavaScript]
    +import {Controller} from "@hotwired/stimulus";
    +
    +
    +export default class extends Controller {
    +
    +  connect() {
    +    this.elementAdded(this.element)
    +  }
    +
    +  elementAdded(element) {
    +    var app;
    +
    +    app = window.App || (window.App = {});
    +    app.tomSelects = {};
    +
    +    app.tomSelects[element.id] = new TomSelect(`#${element.id}`, {
    +      plugins: element.multiple ? ["remove_button"] : [],
    +      create: false,
    +      onItemAdd() {
    +        this.setTextboxValue("");
    +        if(!this.dropdown.classList.contains("single") ) {
    +          this.refreshOptions();
    +        }
    +      },
    +      render: {
    +        no_results() {
    +          // Render localized "no results" message
    +          const message = this.input.dataset.chosenNoResults || "No results found";
    +          return `<div class="no-results">${message}</div>`;
    +        },
    +      }
    +    });
    +  }
    +}    
\end{lstlisting}

\subsection{\_filter.scss}
\begin{lstlisting}[language=JavaScript]
+.w-10 {
+  width: 10%;
+}
+
+.w-90 {
+  width: 90%;
+}
\end{lstlisting}

\newpage

\subsection{\_filter.html.haml}
\begin{lstlisting}[language=JavaScript]
    +- caption ||= t("people_filters.#{type}.title")
    +%div.well
    +  %div.d-flex.flex-row
    +    %h3.w-50.mb-3= caption
    +    - if delete
    +      %div.w-50.d-flex.justify-content-end
    +        = link_to "/groups/#{@group.id}/people_filters/turbo/#{type.to_s}", class: "button btn", data: {turbo_method: :delete} do
    +          %i.fa.fa-trash
    +  = yield
    -- id = type.to_s.pluralize
    -- caption ||= t("people_filters.#{type}.title")
    -- open_class = "show" if entry.filter_chain[type.to_sym].present?
    -
    -.accordion-item
    -  .accordion-header
    -    %a.accordion-button.p-2.bg-white.text-black{ href: "##{id}", data: { bs_toggle: :collapse } }
    -      = caption
    -
    -  .accordion-collapse.collapse{ id: id, class: open_class }
    -    .accordion-body
    -      = yield
\end{lstlisting}

\newpage

\subsection{people\_filters/\_form.html.haml}
\begin{lstlisting}[language=JavaScript]
+= standard_form(path_args(entry), noindent: true, stacked: true) do |form|
+
+  = render 'search_or_save_buttons', f: form
+
+  = render_extensions :form, locals: { f: form }
+  = form.error_messages
+
+  = render 'range', f: form
+
+  .mt-3.mb-3.ml-3#overview
+    - if action_name == 'edit'
+      %div{id: "role-configuration"}
+        - if entry.filter_chain[:role]
+          = render(layout: 'layouts/filter', locals: { type: :role }) do
+            = render 'role', f: form
+
+      - if @qualification_kinds.present?
+        - if entry.filter_chain[:qualification]
+          %div{id: "qualification-configuration"}
+            = render(layout: 'layouts/filter', locals: { type: :qualification }) do
+              = render 'qualification', f: form
+
+      - if entry.filter_chain[:attributes]
+        %div{id: "attributes-configuration"}
+          = render(layout: 'layouts/filter', locals: { type: :attributes }) do
+            = render 'attributes', f: form
+      - if entry.filter_chain[:tag]
+        %div{id: "tag-configuration"}
+          = render(layout: 'layouts/filter', locals: { type: :tag }) do
+            .label-columns
+              = render 'tag'
+
+    - if action_name == 'new'
+      %h3#filter-advice t("people_filters.filter_advice")
+    %div.dropdown.ml-3{"data-controller": "filter-dropdown"}
+      %button.btn.dropdown-toggle{ type: 'button',  "data-bs-toggle": "dropdown", "aria-expanded": "false"}
+        %i.fa.fa-plus
+      %ul.dropdown-menu#filter-criteria-dropdown{"data-filter-dropdown-target": "menu"}
+        - @filter_criteria.each do |filter_criterion|
+          - if entry.filter_chain[filter_criterion.to_sym].nil?
+            = link_to t("people_filters.#{filter_criterion}.title"), "/groups/#{@group.id}/people_filters/#{filter_criterion}",
+            class: "dropdown-item", id: "dropdown-option-#{filter_criterion}", data: {turbo_stream: true}
 
   - if can?(:create, entry)
     %label.required=PeopleFilter.human_attribute_name(:name)
-    = f.input_field :name, placeholder: t('.save_filter_placeholder'), class: 'mb-3 ', required: false
+    = form.input_field :name, placeholder: t('.save_filter_placeholder'), class: 'mb-3 ', required: false
 
-  = render 'search_or_save_buttons', f: f
+  = render 'search_or_save_buttons', f: form   
\end{lstlisting}

\newpage


\subsection{people\_filters/\_form.html.haml}
\begin{lstlisting}[language=JavaScript]
    += standard_form(path_args(entry), noindent: true, stacked: true) do |form|
    +
    +  = render 'search_or_save_buttons', f: form
    +
    +  = render_extensions :form, locals: { f: form }
    +  = form.error_messages
    +
    +  = render 'range', f: form
    +
    +  .mt-3.mb-3.ml-3#overview
    +    - if action_name == 'edit'
    +      %div{id: "role-configuration"}
    +        - if entry.filter_chain[:role]
    +          = render(layout: 'layouts/filter', locals: { type: :role }) do
    +            = render 'role', f: form
    +
    +      - if @qualification_kinds.present?
    +        - if entry.filter_chain[:qualification]
    +          %div{id: "qualification-configuration"}
    +            = render(layout: 'layouts/filter', locals: { type: :qualification }) do
    +              = render 'qualification', f: form
    +
    +      - if entry.filter_chain[:attributes]
    +        %div{id: "attributes-configuration"}
    +          = render(layout: 'layouts/filter', locals: { type: :attributes }) do
    +            = render 'attributes', f: form
    +      - if entry.filter_chain[:tag]
    +        %div{id: "tag-configuration"}
    +          = render(layout: 'layouts/filter', locals: { type: :tag }) do
    +            .label-columns
    +              = render 'tag'
    +
    +    - if action_name == 'new'
    +      %h3#filter-advice Um eine Filterkriterium auszuwaehlen, bitte klicke auf den Hinzufuege-Button
    +    %div.dropdown.ml-3{"data-controller": "filter-dropdown"}
    +      %button.btn.dropdown-toggle{ type: 'button',  "data-bs-toggle": "dropdown", "aria-expanded": "false"}
    +        %i.fa.fa-plus
    +      %ul.dropdown-menu#filter-criteria-dropdown{"data-filter-dropdown-target": "menu"}
    +        - @filter_criteria.each do |filter_criterion|
    +          - if entry.filter_chain[filter_criterion.to_sym].nil?
    +            = link_to t("people_filters.#{filter_criterion}.title"), "/groups/#{@group.id}/people_filters/#{filter_criterion}",
    +            class: "dropdown-item", id: "dropdown-option-#{filter_criterion}", data: {turbo_stream: true}    
\end{lstlisting}

\newpage

\subsection{\_qualification.html.haml}
\begin{lstlisting}[language=JavaScript]
-- filter = entry.filter_chain[:qualification]
-- filter_args = filter ? filter.args : {}
+- filter_qualifications = entry.filter_chain[:qualification]
+- filter_args = filter_qualifications ? filter_qualifications.args : {}
+- present_selected_qualifications = filter_qualifications.present? ? filter_qualifications.to_hash[:qualification_kind_ids] : []
+- qualification_options = options_from_collection_for_select(@qualification_kinds, :second, :first, present_selected_qualifications)
 
-.label-columns
-  = field_set_tag(t('.prompt_qualification_selection')) do
-    - unless can?(:index_full_people, @group)
-      .alert.alert-warning= t('.not_enough_permissions')
+- present_selected_validity = filter_qualifications.present? ? filter_qualifications.to_hash[:validity] : []
+- validity_options = options_from_collection_for_select(@validities, :second, :first, present_selected_validity)
 
-    .controls
-      - @qualification_kinds.each do |kind|
-        - dom_id = "qualification_kind_id_#{kind.id}"
-        = label_tag(dom_id, class: 'checkbox inline') do
-          = check_box_tag("filters[qualification][qualification_kind_ids][]",
-                          kind.id,
-                          filter && filter.args[:qualification_kind_ids].include?(kind.id),
-                          id: dom_id)
-          = kind.to_s
 
+.label-columns.mr-5
+  - unless can?(:index_full_people, @group)
+    .alert.alert-warning= t('.not_enough_permissions')
 
-    .controls
-      &nbsp;
+  .d-flex.flex-row.align-items-center.mb-2
+    .w-10
+      %label= t("people_filters.qualification.title")
+    .w-90.shown{"data-controller": "form-select"}
+      = select_tag('filters[qualification][qualification_kind_ids][]',
+                      qualification_options,
+                      class: 'form-control tom-select w-100',
+                      id: 'qualification-select',
+                      "data-controller": "form-select",
+                      multiple: true,
+                      data: { chosen_no_results: t('global.chosen_no_results'),
+                      placeholder: t("people_filters.qualification.prompt_qualification_placeholder") } )
+  .d-flex.flex-row.align-items-center.mb-2
+    .w-10
+      %label= t("people_filters.qualification.criterion")
+    .w-90.shown.d-flex.flex-column
+      .d-flex.flex-row
+        = label_tag('filters_qualification_match_one', class: 'radio inline') do
+          = radio_button_tag('filters[qualification][match]', 'one', true)
+          = t('people_filters.simple_radio.match.one')
+      .d-flex.flex-row
+        = label_tag('filters_qualification_match_all', class: 'radio inline') do
+          = radio_button_tag('filters[qualification][match]', 'all', filter_args[:match] == 'all', %w(not_active none only_expired).include?(filter_args[:validity]) ? { disabled: true } : {})
+          = t('people_filters.simple_radio.match.all')
\end{lstlisting}

\subsection{\_qualification.html.haml}
\begin{lstlisting}[language=JavaScript]
-    .controls
-      = label_tag('filters_qualification_match_one', class: 'radio inline') do
-        = radio_button_tag('filters[qualification][match]', 'one', true)
-        = t('people_filters.simple_radio.match.one')
-      = label_tag('filters_qualification_match_all', class: 'radio inline') do
-        = radio_button_tag('filters[qualification][match]', 'all', filter_args[:match] == 'all', %w(not_active none only_expired).include?(filter_args[:validity]) ? { disabled: true } : {})
-        = t('people_filters.simple_radio.match.all')
-
-  = field_set_tag(t('.prompt_validity')) do
-    %div.row
-      = render 'simple_radio',
-                attr: "filters[qualification][validity]",
-                value: 'active',
-                checked: true # first item is checked per default
-      = render 'simple_radio',
-                attr: "filters[qualification][validity]",
-                value: 'reactivateable',
-                checked: filter_args[:validity] == 'reactivateable'
-      = render 'simple_radio',
-                attr: "filters[qualification][validity]",
-                value: 'not_active_but_reactivateable',
-                checked: filter_args[:validity] == 'not_active_but_reactivateable'
-      = render 'simple_radio',
-                attr: "filters[qualification][validity]",
-                value: 'not_active',
-                checked: filter_args[:validity] == 'not_active'
-      = render 'simple_radio',
-                attr: "filters[qualification][validity]",
-                value: 'all',
-                checked: filter_args[:validity] == 'all'
-      = render 'simple_radio',
-                attr: "filters[qualification][validity]",
-                value: 'none',
-                checked: filter_args[:validity] == 'none'
-      = render 'simple_radio',
-                attr: "filters[qualification][validity]",
-                value: 'only_expired',
-                checked: filter_args[:validity] == 'only_expired'

-= field_set_tag(t('.prompt_date'), id: 'reference-date', class: 'form-horizontal', style: "display: #{%w(all none).include?(filter_args[:validity]) ? 'none' : 'block'};") do
-  .control-group.row
-    .col-2
-      .input-group
+  .d-flex.flex-row.align-items-center.mb-2
+    .w-10
+      %label= t("people_filters.qualification.validity")
+    .w-90.shown.d-flex.flex-column{"data-controller": "form-select"}
+      = select_tag('filters[qualification][validity]',
+                      validity_options,
+                      class: 'form-control tom-select w-100',
+                      "data-controller": "form-select",
+                      id: 'qualification-validity-select',
+                      multiple: false,
+                      data: { chosen_no_results: t('global.chosen_no_results'),
+                      placeholder: t("people_filters.qualification.prompt_qualification_validity_placeholder") } )
\end{lstlisting}

\newpage

\subsection{\_qualification.html.haml}
\begin{lstlisting}[language=JavaScript]
+  .d-flex.flex-row.align-items-center.mb-2
+    .w-10
+      %label= t('.prompt_date')
+    .w-90.shown.d-flex.flex-row{"data-controller": "form-select"}
+      .input-group.w-75.mr-3
         %span.input-group-text= icon(:'calendar-alt')
         = text_field_tag("filters[qualification][reference_date]", filter_args[:reference_date], class: 'date form-control form-control-sm')
-    .col-10.col-form-label
-      = t('.reference_date_help_inline')
-
-
-= field_set_tag(t('.prompt_year'), id: 'year-scope', class: 'form-horizontal', style: "display: #{filter_args[:validity] == 'all' ? 'block' : 'none'};") do
-  .control-group.row
-    = label_tag(:start_at_year_from, t('.start_at_year_label'), class: 'control-label col-form-label col-2')
-    .col-2
-      = number_field_tag("filters[qualification][start_at_year_from]", filter_args[:start_at_year_from], class: 'form-control form-control-sm', placeholder: 'YYYY' )
-    .col-1.col-form-label
-      = t('.start_at_year_infix')
-    .col-2
-      = number_field_tag("filters[qualification][start_at_year_until]", filter_args[:start_at_year_until], class: 'form-control form-control-sm', placeholder: 'YYYY')
-  .control-group.row
-    = label_tag(:finish_at_year_from, t('.finish_at_year_label'), class: 'control-label col-form-label col-2')
-    .col-2
-      = number_field_tag("filters[qualification][finish_at_year_from]", filter_args[:finish_at_year_from], class: 'form-control form-control-sm', placeholder: 'YYYY')
-    .col-1.col-form-label
-      = t('.finish_at_year_infix')
-    .col-2
-      = number_field_tag("filters[qualification][finish_at_year_until]", filter_args[:finish_at_year_until], class: 'form-control form-control-sm', placeholder: 'YYYY'  )
+      .w-25.col-form-label
+        = t('.reference_date_help_inline')
\end{lstlisting}

\newpage

\subsection{\_role.html.haml}
\begin{lstlisting}[language=JavaScript]
-- filter = entry.filter_chain[:role]
+- filter_roles = entry.filter_chain[:role]
+- present_selected_roles = filter_roles.present? ? filter_roles.to_hash[:role_type_ids] : []
+- role_options = options_from_collection_for_select(@roles, :second, :first, present_selected_roles)
 
-.label-columns
-  = field_set_tag(t('.prompt_role_selection')) do
-    - @role_types.each do |layer, groups|
-      .layer{ class: [@group.klass.label, @group.layer_group.class.label].include?(layer) && 'same-layer' }
-        %h4.filter-toggle= layer
-        - groups.each do |group, role_types|
-          .group.control-group{ class: group == @group.klass.label && 'same-group' }
-            %h5.filter-toggle= group
-            .controls
-              - role_types.each do |role_type|
-                - id = "filters_role_role_type_ids_#{role_type.id}"
-                = label_tag(nil, id, class: 'checkbox inline') do
-                  = check_box_tag("filters[role][role_type_ids][]",
-                                  role_type.id,
-                                  filter && filter.to_hash[:role_types].include?(role_type.to_s),
-                                  id: id)
-                  = role_type.label
+- present_selected_kinds = filter_roles.present? ? filter_roles.to_hash[:kind] : []
+- kind_options = options_from_collection_for_select(@kinds, :second, :first, present_selected_kinds)
 
-  = field_set_tag(t('.prompt_role_duration_selection')) do
-    .row.control-group.mt-3
-      = render 'simple_radio',
-               attr: "filters[role][kind]",
-               value: 'active_today',
-               checked: !filter || !Person::Filter::Role::KINDS.include?(filter.args[:kind])
-    .row.control-group.mt-2
-      .d-flex.col-xl-6
-        %label.mt-1.nowrap.w-50=t('.prompt_role_duration_from')
-        .input-group.w-50
-          %span.input-group-text= icon(:'calendar-alt')
-          = text_field_tag("filters[role][start_at]", filter && filter.args[:start_at], class: 'date col-2 form-control form-control-sm')
-        %label.mx-2.mt-1.w-50=t('.prompt_role_duration_until')
-        .input-group.w-50
-          %span.input-group-text= icon(:'calendar-alt')
-          = text_field_tag("filters[role][finish_at]", filter && filter.args[:finish_at], class: 'date col-2 form-control form-control-sm')
-    .row.control-group.mt-2
-      - Person::Filter::Role::KINDS.each do |key|
-        = render 'simple_radio',
-            attr: "filters[role][kind]",
-            value: key,
-            checked: filter && filter.args[:kind] == key
-    .row.control-group.mt-3
-      = label_tag(nil, 'filters_role_include_archived', class: 'checkbox inline mt-2') do
-        = check_box_tag('filters[role][include_archived]', true, filter && true?(filter.args[:include_archived]))
-        = t('.include_archived')

\end{lstlisting}

\newpage

\subsection{\_role.html.haml}
\begin{lstlisting}[language=JavaScript]
+.label-columns.mr-5
+  .d-flex.flex-row.align-items-center.mb-2
+    .w-10
+      %label= t("people_filters.role.title")
+    .w-90
+      = select_tag('filters[role][role_type_ids][]',
+                        role_options,
+                        class: 'form-control tom-select w-100',
+                        "data-controller": "form-select",
+                        id: 'role-select',
+                        multiple: true,
+                        data: { chosen_no_results: t('global.chosen_no_results'), placeholder: t('.prompt_role_placeholder') } )
+  .d-flex.flex-row.align-items-center.control-group.mb-2
+    .w-10
+      %label= t("people_filters.role.duration_title")
+    .w-90.d-flex.flex-row.align-items-center
+      .input-group.w-50
+        %span.input-group-text= icon(:'calendar-alt')
+        = text_field_tag("filters[role][start_at]", filter_roles && filter_roles.args[:start_at], class: 'date form-control')
+        %label.mx-2.mt-1=t('.prompt_role_duration_until')
+      .input-group.w-50
+        %span.input-group-text= icon(:'calendar-alt')
+        = text_field_tag("filters[role][finish_at]", filter_roles && filter_roles.args[:finish_at], class: 'date form-control')
+  .d-flex.flex-row.align-items-center
+    .w-10
+      %label= t("people_filters.role.role_kind_title")
+    .w-90{"data-controller": "form-select"}
+      = select_tag('filters[role][kind][]',
+                        kind_options,
+                        class: 'form-control tom-select w-100',
+                        "data-controller": "form-select",
+                        id: 'role-kind-select',
+                        multiple: true,
+                        data: { chosen_no_results: t('global.chosen_no_results'), placeholder: t('.prompt_role_kind_placeholder') } )

\end{lstlisting}

\newpage

\subsection{\_tag.html.haml}
\begin{lstlisting}[language=JavaScript]
-- filter = entry.filter_chain[:tag]
-- selected = filter ? filter.to_hash[:names] : []
-- options = options_from_collection_for_select(@possible_tags, :second, :first, selected)
+- present_filter_tags = entry.filter_chain[:tag]
+- present_selected = present_filter_tags ? present_filter_tags.to_hash[:names] : []
+- present_options = options_from_collection_for_select(@possible_tags, :second, :first, present_selected)
 
-%p= t('.prompt_tag_selection')
+- absent_filter = entry.filter_chain[:tag_absence]
+- absent_selected = absent_filter ? absent_filter.to_hash[:names] : []
+- absent_options = options_from_collection_for_select(@possible_tags, :second, :first, absent_selected)
 
-.shown
-  = select_tag('filters[tag][names][]',
-                options,
-                class: 'form-control tom-select',
-                multiple: true,
-                data: { chosen_no_results: t('global.chosen_no_results'), placeholder: t('.prompt_tag_placeholder') } )
+.d-flex.flex-column
+  .d-flex.flex-row.align-items-center.mb-2
+    .w-10
+      %label= t('.prompt_tag_selection')
+    .w-90.shown
+      = select_tag('filters[tag][names][]',
+                    present_options,
+                    class: 'form-control tom-select w-100',
+                    "data-controller": "form-select",
+                    id: 'present-tag-select',
+                    multiple: true,
+                    data: { chosen_no_results: t('global.chosen_no_results'), placeholder: t('.prompt_tag_placeholder') } )
+  .d-flex.flex-row.align-items-center
+    .w-10
+      %label= t('.prompt_tag_absence_selection')
+    .w-90.shown
+      = select_tag('filters[tag_absence][names][]',
+                    absent_options,
+                    class: 'form-control tom-select w-100',
+                    id: 'absent-tag-select',
+                    "data-controller": "form-select",
+                    multiple: true,
+                    data: { chosen_no_results: t('global.chosen_no_results'), placeholder: t('.prompt_tag_placeholder') } )
\end{lstlisting}

\newpage

\subsection{create.turbo\_stream.haml}
\begin{lstlisting}[language=JavaScript]
+= turbo_stream.prepend "overview" do
+  %div{id: "#{@filter_criterion}-configuration"}
+    = render(layout: 'layouts/filter', locals: { type: @filter_criterion.to_sym, delete: true}) do
+      = render @filter_criterion
+
+= turbo_stream.remove "dropdown-option-#{@filter_criterion}"
+= turbo_stream.remove "filter-advice"
\end{lstlisting}

\subsection{delete.turbo\_stream.haml}
\begin{lstlisting}[language=JavaScript]
+= turbo_stream.remove "#{@filter_criterion}-configuration"
+
+= turbo_stream.append "filter-criteria-dropdown" do
+  = link_to t("people_filters.#{@filter_criterion}.title"), "/groups/#{@group.id}/people_filters/#{@filter_criterion}",
+          class: "dropdown-item", id: "dropdown-option-#{@filter_criterion}", data: {turbo_stream: true}
\end{lstlisting}

\subsection{subscriber/\_form.html.haml}
\begin{lstlisting}[language=JavaScript]
-  .accordion
-    = render(layout: 'people_filters/filter', locals: { entry: @mailing_list, type: :attributes }) do
-      = render 'people_filters/attributes', f: f, entry: @mailing_list
-
-    - FeatureGate.if('person_language') do
-      = render(layout: 'people_filters/filter', locals: { type: :language }) do
-        = render 'people_filters/language', f: f, entry: @mailing_list
-
-    = render_extensions('filter_form', locals: { entry: entry })
+  = render(layout: 'layouts/filter', locals: { entry: @mailing_list, type: :attributes, delete: false}) do
+    = render 'people_filters/attributes', f: f, entry: @mailing_list
+  = render_extensions('filter_form', locals: { entry: entry })
\end{lstlisting}

\subsection{routes.rb}
\begin{lstlisting}[language=Ruby]
+      # Turbo Routes
+      get 'people_filters/:filter_criterion' => 'people_filters#filter_criterion'
+      delete 'people_filters/turbo/:filter_criterion' => 'people_filters#filter_criterion'
\end{lstlisting}

\newpage

\subsection{global\_conditions\_spec.rb}
\begin{lstlisting}[language=Ruby]
+# frozen_string_literal: true
+
+#  Copyright (c) 2024, Schweizer Alpen-Club. This file is part of
+#  hitobito_sac_cas and licensed under the Affero General Public License version 3
+#  or later. See the COPYING file at the top-level directory or at
+#  https://github.com/hitobito/hitobito_sac_cas.
+
+require "spec_helper"
+
+describe "Global Conditions", js: true do
+  let(:lang) { :de }
+  let(:top_leader) { people(:top_leader) }
+  let(:alice) { people(:bottom_member) }
+
+  context "modify" do
+    let(:path) {
+      edit_group_mailing_list_filter_path(group_id: MailingList.first.group_id,
+        mailing_list_id: MailingList.first.id)
+    }
+
+    before {
+      sign_in(top_leader)
+      visit path
+    }
+
+    it "can add global conditions" do
+      select "Firmenname", from: "attribute_filter"
+      first(".attribute_constraint_dropdown").find("option[value='not_match']").select_option
+      first("input.form-control[type='text']").set("Puzzle ITC")
+
+      select "Alter", from: "attribute_filter"
+      all(".attribute_constraint_dropdown")[1].find("option[value='equal']").select_option
+      all("input.form-control[type='text']")[1].set(56)
+
+      first(".btn.btn-primary", text: "Speichern").click
+
+      within("div#main") do
+        expect(page).to have_text("Firmenname enthaelt nicht Puzzle ITC")
+        expect(page).to have_text("Alter ist genau 56")
+      end
+    end
+
+    context "member access" do
+      before {
+        Rails.env.stub(production?: true)
+        sign_in(alice)
+      }
+
+      it "can't access global conditions" do
+        visit path
+        expect(page).to have_selector("div.alert-danger", text: "Sie sind nicht berechtigt, diese Seite anzuzeigen")
+      end
+    end
+  end
+end
\end{lstlisting}

\newpage

\subsection{people\_filter\_spec.rb}
\begin{lstlisting}[language=Ruby]
    +# frozen_string_literal: true
+
+#  Copyright (c) 2024, Schweizer Alpen-Club. This file is part of
+#  hitobito_sac_cas and licensed under the Affero General Public License version 3
+#  or later. See the COPYING file at the top-level directory or at
+#  https://github.com/hitobito/hitobito_sac_cas.
+
+require "spec_helper"
+
+describe "PeopleFilter", js: true do
+  let(:lang) { :de }
+  let(:top_leader) { people(:top_leader) }
+  let(:alice) { people(:bottom_member) }
+
+  context "filtering" do
+    let(:path) { new_group_people_filter_path(group_id: Group.first.id) }
+
+    before {
+      sign_in(top_leader)
+      visit path
+      find(".btn.dropdown-toggle").click
+    }
+
+    context "tags" do
+      before {
+        alice.tag_list.add("lorem")
+        alice.tag_list.add("ipsum")
+        alice.save!
+        find("#dropdown-option-tag").click
+      }
+
+      it "can filter by present tags" do
+        expect(page).to have_selector("div#tag-configuration")
+        find("#present-tag-select-ts-control").set("lorem")
+        find("#present-tag-select-opt-2").click
+
+        first(".btn.btn-primary", text: "Suchen").click
+        within(".table.table-striped.table-hover") do
+          # Head table row and table row of found user
+          expect(all("tr").size).to eq(2)
+          expect(page).to have_selector("#person_#{alice.id}")
+        end
+      end
+
+      it "can filter by absent tags" do
+        visit path
+        find(".btn.dropdown-toggle").click
+        find("#dropdown-option-tag").click
+        expect(page).to have_selector("div#tag-configuration")
+        find("#absent-tag-select-ts-control").set("ipsum")
+        find("#absent-tag-select-opt-1").click
+
+        first(".btn.btn-primary", text: "Suchen").click
+        expect(page).to have_no_selector(".table.table-striped.table-hover")
+        expect(page).to have_no_selector("#person_#{alice.id}")
+      end
+    end
\end{lstlisting}

\newpage

\subsection{people\_filter\_spec.rb}
\begin{lstlisting}[language=Ruby]
+    context "roles" do
+      before {
+        find("#dropdown-option-role").click
+      }
+
+      it "can filter by present roles" do
+        expect(page).to have_selector("div#role-configuration")
+        # Select role
+        find("#role-select-ts-control").set(alice.roles.first.label)
+        find("#role-select-opt-2").click
+
+        # Select date
+        find("#filters_role_start_at").set("2025-03-18")
+        find("#filters_role_finish_at").set("2025-03-30")
+
+        # Select role kind
+        find("#role-kind-select-ts-control").set("Aktiv")
+        find("#role-kind-select-opt-1").click
+
+        first(".btn.btn-primary", text: "Suchen").click
+        within(".table.table-striped.table-hover") do
+          # Head table row and table row of found user
+          expect(all("tr").size).to eq(2)
+          expect(page).to have_selector("#person_#{alice.id}")
+        end
+      end
+    end
+
+    context "qualification" do
+      before {
+        find("#dropdown-option-qualification").click
+      }
+
+      it "can filter by qualifications" do
+        expect(page).to have_selector("div#qualification-configuration")
+
+        # Select qualification
+        find("#qualification-select-ts-control")
+          .set(alice.qualifications.first.qualification_kind.label)
+        find("#qualification-select-opt-4").click
+
+        # Select reference date
+        find("#filters_qualification_reference_date").set("2025-03-05")
+
+        first(".btn.btn-primary", text: "Suchen").click
+        within(".table.table-striped.table-hover") do
+          # Head table row and table row of found user
+          expect(all("tr").size).to eq(2)
+          expect(page).to have_selector("#person_#{alice.id}")
+        end
+      end
+    end
\end{lstlisting}

\newpage

\subsection{people\_filter\_spec.rb}
\begin{lstlisting}[language=Ruby]
    +    context "attributes" do
    +      before {
    +        find("#dropdown-option-attributes").click
    +      }
    +
    +      it "can filter by attributes" do
    +        expect(page).to have_selector("div#attributes-configuration")
    +
    +        select "PLZ", from: "attribute_filter"
    +        first(".attribute_constraint_dropdown").find("option[value='equal']").select_option
    +        first("input.form-control[type='text']").set(alice.zip_code)
    +
    +        select "Vorname", from: "attribute_filter"
    +        all(".attribute_constraint_dropdown")[1].find("option[value='equal']").select_option
    +        all("input.form-control[type='text']")[1].set(alice.first_name)
    +
    +        first(".btn.btn-primary", text: "Suchen").click
    +        within(".table.table-striped.table-hover") do
    +          # Head table row and table row of found user
    +          expect(all("tr").size).to eq(2)
    +          expect(page).to have_selector("#person_#{alice.id}")
    +        end
    +      end
    +
    +      it "is xss-attack immune" do
    +        expect(page).to have_selector("div#attributes-configuration")
    +
    +        select "Ort", from: "attribute_filter"
    +        first(".attribute_constraint_dropdown").find("option[value='match']").select_option
    +        first("input.form-control[type='text']").set("<script>alert('Hacked!');</script>")
    +
    +        first(".btn.btn-primary", text: "Suchen").click
    +        expect {
    +          page.driver.browser.switch_to.alert
    +        }.to raise_error(Selenium::WebDriver::Error::NoSuchAlertError)
    +      end
    +    end
\end{lstlisting}

\newpage

\subsection{people\_filter\_spec.rb}
\begin{lstlisting}[language=Ruby]
+    context "saving" do
+      let(:path) { new_group_people_filter_path(group_id: Group.first.id) }
+
+      before {
+        sign_in(top_leader)
+        visit path
+        find(".btn.dropdown-toggle").click
+      }
+
+      it "can save filter" do
+        find("#dropdown-option-role").click
+
+        find("#role-select-ts-control").click
+        find("#role-select-opt-1").click
+
+        filter_name = "Filtername"
+        fill_in "people_filter_name", with: filter_name
+        first(".btn.btn-primary", text: "Suche speichern").click
+
+        expect(page).to have_selector("div.alert-success", text: "Filter #{filter_name} wurde erfolgreich erstellt.")
+      end
+    end
+  end     

+  context "member access" do
+    let(:path) { new_group_people_filter_path(group_id: Group.find_by(name: "Top").id) }
+
+    before {
+      Rails.env.stub(production?: true)
+      sign_in(alice)
+    }
+
+    it "can't access people filtering" do
+      visit path
+      expect(page).to have_selector("div.alert-danger", text: "Sie sind nicht berechtigt, diese Seite anzuzeigen")
+    end
+  end
+end    
\end{lstlisting}

\subsection{role.rb}
\begin{lstlisting}[language=Ruby]
def to_hash
+    merge_duration_args(role_types: args[:role_types], role_type_ids: args[:role_type_ids], kind: args[:kind])
end
\end{lstlisting}

\newpage

\subsection{\_main.scss}
\begin{lstlisting}[language=Ruby]
def to_hash
+    merge_duration_args(role_types: args[:role_types], role_type_ids: args[:role_type_ids], kind: args[:kind])
end
\end{lstlisting}

\subsection{qualifications.yml}
\begin{lstlisting}[language=Ruby]
+ bottom_member:
+    person: bottom_member
+    origin: "Verwalter"
+    qualification_kind: sl
+    start_at: 2025-03-04
+    finish_at: 2025-03-30
\end{lstlisting}

\subsection{views.fr.yml}
\begin{lstlisting}[language=Ruby]
people_filters:
+    filters_role_kind:
+        active_today: Actuellement actif
+        active: Actif
+        created: Cree
+        deleted: Supprime
+        inactive: Inactif

tag:
+    prompt_tag_selection: Inclure
+    prompt_tag_absence_selection: Exclure

role:
+    role_kind_title: Validite
+    prompt_role_placeholder: Saisir le nom du role souhaite
+    prompt_role_kind_placeholder: Saisir la validite souhaitee
+    duration_title: Date de reference

qualification:
+    criterion: Critere
+    validity: Validite
+    validity_label:
+      active: Actif
+      reactivateable: Reactivable
+      not_active_but_reactivateable: Inactif mais reactivable
+      not_active: Non actif
+      all: Tous
+      none: Personne
+      only_expired: Uniquement les produits expires
+    prompt_qualification_placeholder: Saisir le nom de qualification souhaite
+    prompt_qualification_validity_placeholder: Saisir la validite souhaitee
\end{lstlisting}

\newpage

\subsection{views.it.yml}
\begin{lstlisting}[language=Ruby]
people_filters:
+    filters_role_kind:
+        active: Attualmente attivo
+        created: Creato
+        deleted: Annullato
+        inactive: Inattivo

tag:
+    prompt_tag_selection: Includere
+    prompt_tag_absence_selection: Escludere

role:
+    role_kind_title: Validita
+    prompt_role_placeholder: Inserire il nome del ruolo desiderato
+    prompt_role_kind_placeholder: Inserire la validita desiderata
+    duration_title: Data chiave

qualification:
+    criterion: Criterio
+    validity: Validita
+    validity_label:
+      active: Attivo
+      reactivateable: Riattivabile
+      not_active_but_reactivateable: Non e attivo ma puo essere riattivato
+      not_active: Non attivo
+      all: Tutti
+      none: Nessuno
+      only_expired: Solo scaduto
+    prompt_qualification_placeholder: Inserire il nome della qualifica desiderata
+    prompt_qualification_validity_placeholder: Inserire la validita desiderata
\end{lstlisting}

\newpage

\subsection{views.de.yml}
\begin{lstlisting}[language=Ruby]
people_filters:
+    filters_role_kind:
+        active: Aktiv
+        created: Erstellt
+        deleted: Geloescht
+        inactive: Inaktiv

tag:
+    prompt_tag_selection: Einschliessen
+    prompt_tag_absence_selection: Ausschliessen

role:
+    role_kind_title: Gueltigkeit
+    prompt_role_placeholder: Gewuenschter Rollenname eingeben
+    prompt_role_kind_placeholder: Gewuenschte Gueltigkeit eingeben
+    duration_title: Stichdatum

qualification:
+    criterion: Kriterium
+    validity: Gueltigkeit
+    validity_label:
+      active: Aktiv
+      reactivateable: Reactivable
+      not_active_but_reactivateable: Nicht aktiv aber reaktivierbar
+      not_active: Nicht aktiv
+      all: Alle
+      none: Keine
+      only_expired: Nur abgelaufene
+    prompt_qualification_placeholder: Gewuenschter Qualifikationsname
+    prompt_qualification_validity_placeholder: Gewuenschte Gueltigkeit
\end{lstlisting}

\newpage

\subsection{views.en.yml}
\begin{lstlisting}[language=Ruby]
people_filters:
+    filters_role_kind:
+        active: Active
+        created: Created
+        deleted: Deleted
+        inactive: Inactive

tag:
+    prompt_tag_selection: Include
+    prompt_tag_absence_selection: Exclude

role:
+    role_kind_title: Validity
+    prompt_role_placeholder: Enter the desired role name
+    prompt_role_kind_placeholder: Enter the desired validity
+    duration_title: Key date

qualification:
+    criterion: Criterion
+    validity: Validity
+    validity_label:
+      active: Active
+      reactivateable: Reactivateable
+      not_active_but_reactivateable: Not active but reactivateable
+      not_active: Not active
+      all: All
+      none: None
+      only_expired: Oly expired
+    prompt_qualification_placeholder: Enter desired qualification name
+    prompt_qualification_validity_placeholder: Enter the desired validity
\end{lstlisting}

\newpage

\section{Nachweis Datensicherung}

\subsection{Dokumentationssicherung mit Git}
\begin{figure}[h]
    \centering
    \fbox{\includegraphics[width=0.8\textwidth,]{nachweis_dok_git.png}}
    \caption{Datensicherung Code Repository}
 \end{figure}

 \newpage

\subsection{Codesicherung mit Git}
 \begin{figure}[h]
    \centering
    \fbox{\includegraphics[width=0.8\textwidth,]{nachweis_code_git.png}}
    \caption{Datensicherung Code Repository}
 \end{figure}

 \subsection{USB-Sicherung Dokumentation und Code}
 \begin{figure}[h]
    \centering
    \fbox{\includegraphics[width=0.8\textwidth,]{nachweis_dok_usb.png}}
    \caption{Datensicherung Dokumentation und Code USB-Stick}
 \end{figure}